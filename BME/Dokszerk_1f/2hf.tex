%Egy .tex fájlba írd meg az alábbihoz hasonló verset a tanult környezetek segítségével:

%Megy a juhász a szamáron,
%Földig ér a lába;
%Nagy a legény, de nagyobb
%Boldogtalansága

%Gyakorold a felsorolásokat (enumerate, itemize, description stb.) és formázásukat ilyesmi példákkal:
% 1. Emlősök
%	• Házi állatok,
% 	  1. macska 
%     2. kutya
%	• Vadon élő állatok,
%	  o farkas, 
%     o oroszlán
% 2. Hüllők
% 	 • kígyó,
% 	 • gyík

\documentclass[a4paper]{article}
\usepackage[utf8]{inputenc}
\usepackage{t1enc}
\usepackage[magyar]{babel}
\usepackage{paralist}

\begin{document}

\begin{verse}
Megy a juhász szamáron,\\
Földig ér a lába;\\
Nagy a legény, de nagyobb\\
Boldogtalansága.
\dots
\end{verse}

\begin{compactenum}
\renewcommand{\labelenumii}{\arabic{enumii}.}
\begin{enumerate}
\item Emlősők
	\begin{itemize}
	\item Házi állatok
		\begin{enumerate}
		\item macska
		\item kutya
		\end{enumerate}
	\item Vadon élő állatok
		\begin{itemize}
		\item[$\circ$] farkas
		\item[$\circ$] oroszlán
		\end{itemize}
	\end{itemize}
\item Hüllők
	\begin{itemize}
	\item farkas
	\item oroszlán
	\end{itemize}
\end{enumerate}
\end{compactenum}

\end{document}