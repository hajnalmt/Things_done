\documentclass[a4paper]{article}
\usepackage[utf8]{inputenc}
\usepackage{t1enc}
\usepackage[magyar]{babel}
\usepackage{array}
\usepackage{multirow}
\usepackage{booktabs}

\begin{document}

\begin{tabbing}
Elment a szóköz \= arrébb\\
\> ez meg még beljebb \= ment\\
\> \> végül \\
Ez lett \=\\
\> de ez van \> így jártunk!
\end{tabbing}

\begin{tabular}{|lrc|}
Ez egy balra igazított oszlop & ez meg jobbra igazított oszlop \\
akkor látszik & ha több elem
\end{tabular}

\begin{tabular}{|r|l|}
\hline
Egy & megérett a meggy \\
\hline
Kettő & csipkebokor vessző \\
\hline
\end{tabular} 

\begin{tabular}{|l||*8{c|}}
\hline
A vonat száma: & 437 & ... \\
\hline\hline
Indul Bp.\ Keleti-pu.-ról & 8:00 & ... \\
Érkezik Hatvanba & 8:52 & ... \\
Indul Hatvanból & 8:55 & ... \\
Érkezik Miskolc Tiszai-pu.-ra & 10:22 & ... \\
\hline
\end{tabular}


\begin{tabular}{|m{2cm}|m{4cm}|m{1cm}|b{3cm}|}
\hline
Éljen a fix méretű tábla! & Éljen a fix méretű
tábla!&
Éljen a fix méretű tábla!& Éljen a fix méretű
tábla! \\
\hline
\end{tabular}

\begin{tabular}{ l| b{2cm}| m{2cm}| p{2cm}}
\hline
középre igazítva & tetejére igazítva &
középre igazítva & cella aljára igazítva\\
\hline
\end{tabular}

\begin{tabular}{|l|c|}
\hline\multirow{2}*{Alma} & 1\\
& 2\\\hline
\multirow{3}{3cm}{3\,cm széles szöveg törve} &
3\\\cline{2-2}
& 4\\\cline{2-2}
& 5\\\hline
\end{tabular}

\begin{tabular}{l||*8{r@{:}l|}}
A vonat száma: & \multicolumn{2}{c|}{437} & ...
\\
\hline\hline
Indul Bp.\ Keleti-pu.-ról & 8&00 & ... \\
Érkezik Hatvanba & 8&52 & ... \\
Indul Hatvanból & 8&55 & ... \\
Érkezik Miskolc Tiszai-pu.-ra & 10&22 & ... \\
\hline
\end{tabular}

\begin{tabular}{c|r@{,}l}
Kifejezés & \multicolumn{2}{c}{Érték} \\
\hline
$\pi$ & 3&1415927 \\
$\pi^\pi$ & 36&46216 \\
$\pi^{\pi^\pi}$ & 80662&666
\end{tabular}

\begin{tabular}{@{}lrr@{}}
\toprule
&\multicolumn{2}{c}{Év}\\
\cmidrule{2-3}
& \multicolumn{1}{c}{2002} &
\multicolumn{1}{c}{2003}\\
\midrule
Jövedelem (Ft)& 775\,000 & 1\,166\,500\\
Adó (Ft) & 165\,000 & 194\,950\\
\bottomrule
\end{tabular}

\hrulefill \begin{tabular}[t]{c} 1 \\ 2 \\ 3 \\ 4\end{tabular}%
\hrulefill \begin{tabular}{c} 1 \\ 2 \\ 3 \\ 4 \end{tabular}%
\hrulefill \begin{tabular}[b]{c} 1 \\ 2 \\ 3 \\ 4 \end{tabular}%
\hrulefill \\

\begin{table}[htb]
\caption{Nagy bölcsességek} %így lehet
címet adni
\label{tab:fontos}
\center
\begin{tabular}{c|r@{,}l}
Kifejezés & \multicolumn{2}{c}{Érték} \\
\hline
$\pi$ & 3&1415927 \\
$\pi^\pi$ & 36&46216 \\
$\pi^{\pi^\pi}$ & 80662&666
\end{tabular}
\end{table}

\end{document}