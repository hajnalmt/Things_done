\documentclass[a4paper]{article}
\usepackage[utf8]{inputenc}
\usepackage{t1enc}
\usepackage[magyar]{babel}

\sloppy

\begin{document}

\begin{tabbing}
Elment a szóköz \= arrébb\\
\> ez meg még beljebb \= ment\\
\> \> végül \\
Ez lett \=\\
\> de ez van \> így jártunk!
\end{tabbing}

\begin{tabular}{|lrc|}
Ez egy balra igazított oszlop & ez meg jobbra igazított oszlop
\\
akkor látszik & ha több elem
\end{tabular}

\begin{tabular}{|r|l|}
\hline
Egy & megérett a meggy \\
\hline
Kettő & csipkebokor vessző \\
\hline
\end{tabular} 

\begin{tabular}{|l||*8{c|}}
\hline
A vonat száma: & 437 & ... \\
\hline\hline
Indul Bp.\ Keleti-pu.-ról & 8:00 & ... \\
Érkezik Hatvanba & 8:52 & ... \\
Indul Hatvanból & 8:55 & ... \\
Érkezik Miskolc Tiszai-pu.-ra & 10:22 & ... \\
\hline
\end{tabular}

\end{document}