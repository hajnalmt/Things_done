%----------------------------------------------------------------------------
\chapter{Elképzeléseink}\label{sect:Visions}
%----------------------------------------------------------------------------
\hspace{2mm} A megfelelő feeling megtalálása, valamint az ELTE PPK-val, a MŰHASZ-szal, a Szponzorokkal, a kisfőnökökkel és a körökkel való kapcsolattartás a kardinális pontjai szerintünk a szervezésnek.

%----------------------------------------------------------------------------
\section{Feeling}
%----------------------------------------------------------------------------
\hspace{2mm} Célunk egy olyan feeling megtalálása, melybe minden rendező bele tudja élni magát és amit a kisfőnökök is át tudnak érezni. Ez jelentheti a közös munkánk alapkövét, így szeretnénk a kisfőnökökkel közösen megszavaztatni, megvitatni. Természetesen vannak már ötleteink:
\begin{itemize}
	\item Broadway: ez a New York-i tőzsde és színház világa
	\item Maszkos bál: a 2012-es Gólyabálunk csodás a velencei karnevált megidéző feelingje 
	\item Alíz Csodaországban: a 2011-es Gólyabálunk kirobbanó sikerű feelingje, mely a most kijövő film miatt aktuális is
	\item Maffia: 1970-es évek maffia pillanatai
	\item Párizs: Párizs eleganciájába, az Eiffel-torony és a Moulin Rouge varázslatos világába elkalauzoló feeling
\end{itemize}
\hspace{2mm} Az ötletek közül az utolsót szeretnénk arculatként felhasználni, mivel stílusa elég divatos, de ez csak jelen pályázat esztétikumát növeli, nem döntés, vagy preferáció.

%----------------------------------------------------------------------------
\section{ELTE PPK}
%----------------------------------------------------------------------------
\hspace{2mm} Úgy gondoljuk, hogy az ELTE PPK-val való rendezés minden évben rengeteg tapasztalatot ad a kisfőnököknek, továbbá a két kar közötti jó viszony ápolásának egy nagyszerű pontja a közös Gólyabál, így ezt a már-már hagyományt szeretnénk megtartani. Ebben az évben valószínűleg már jelentős anyagi részben is hozzá fognak tudni járulni a Bálhoz. (Az új helyszínre való tekintettel pedig biztosan, lsd. következő fejezet) Idén lehet kicsit nehezebb dolgunk lesz velük, ugyanis a mostani animátor-koordinátorral Farkas-Kovách Rékával és az RSZB (Rendezvény Szervezési Bizottság) jelenlegi elnökével Lanku Mátéval sem dolgoztunk még együtt, fiatalok és újak a posztjukon. Reméljük az együttműködés a lehető legzökkenőmentesebb lesz, igyekeznénk minél több időt fordítani rájuk.
\begin{wrapfigure}[10]{r}{0.23\textwidth} 
\begin{center}
\includegraphics[width=0.20\textwidth, keepaspectratio]{figures/fekete_no.jpg}
\end{center}
\caption{Fekete hölgy} 
\label{fig:FeketeNo}
\end{wrapfigure}
%----------------------------------------------------------------------------
\section{Szponzorcsoport}
%----------------------------------------------------------------------------
\hspace{2mm} A Bál szponzorációja minden évben sok kérdést vet fel, nem tudni pontosan, hogy mit kell az SVK-n keresztül intézni. A velük és az egyéb szponzorokkal való kapcsolattartás a tombolás kisfőnök feladata lesz a Gólyabál során. Természetesen az ő munkáját szeretnénk minél jobban segíteni, a tárgyalásokon szeretnénk ha legalább egyikünk ott lenne. Berta Lilla a jelenlegi SVK vezetője segítőkész a témában, a vele való jó kapcsolatra különösen odafigyelnénk.

%----------------------------------------------------------------------------
\section{MŰHASZ}
%----------------------------------------------------------------------------
\hspace{2mm} A MŰHASZ számunkra regeteg segítséget jelent a Gólyabálon. A költségeink jó részét állják és a fellépőket olcsóbban tudják megszerezni. A hátulütőjük, hogy Orbán Balázzsal és Czikó Norbival sokszor nehéz a tárgyalás. Olyan félrecsúszások, rendszeresen előfordulnak, mint Peti és Barbi esetében, hogy megmondták, hogy nem lehetnek Fütyülős lányok a Bál-on, erre Norbi hozott Berentzeneseket. Az ilyen kérdéseket szeretnénk korán és a legkorrektebben tisztázni velük. Barbit és Petit megkérdezve, a fellépők tekintetében a főrendezőknek elég szabad kezük van, így ha esetleg úgy adódna mi is egy Halott Pénz-Ivan and the Parasol párost hívnánk el, esetleg Szabó Balázs Band-et.

%----------------------------------------------------------------------------
\section{Kisfőnökök \& Körök}
%----------------------------------------------------------------------------
\hspace{2mm} Fontosnak tartjuk a kisfőnökökkel való folyamatos kommunikációt, ami legjobban szerintünk egy legalább kétheti rövid találkozóban nyilvánul meg, Gólyabál előtti héten pedig többől is. Hasonlóan szeretnénk minden meghívott kör körvezetőjével is találkozni legalább egyszer, ahol beszélünk velünk pár részletről. Ahhoz, hogy elkerüljük a repiken a civódást, így idén nem szeretnénk a Gólyabálon alkoholos italt adni repinek, és ezt szeretnénk majd tisztázni előre a körvezetőkkel. Természetesen köszönetünk jeléül, ha alkoholos formában kértek repit, akkor azt Gólyabál után megvesszük és odaadjuk nekik. Mindketten úgy érezzük, hogy a körökkel való kapcsolattartás egy főrendezői feladat, így ha úgy adódik, szeretnénk ezt a kezünkben tartani és tudásunkhoz mérten legjobban ellátni.