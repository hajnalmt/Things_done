\documentclass[11pt,a4paper,oneside]{report}             % Single-side
%\documentclass[11pt,a4paper,twoside,openright]{report}  % Duplex

%\PassOptionsToPackage{chapternumber=Huordinal}{magyar.ldf}
\usepackage{t1enc}
\usepackage[utf8]{inputenc}
\usepackage[english,magyar]{babel}
\usepackage{amsmath}
\usepackage{amssymb}
\usepackage{enumerate}
\usepackage[thmmarks]{ntheorem}
\usepackage{graphics}
\usepackage{epsfig}
\usepackage{listings}
\usepackage{color}
%\usepackage{fancyhdr}
\usepackage{lastpage}
\usepackage{anysize}
\usepackage{sectsty}
\usepackage{setspace}  % Ettol a tablazatok, abrak, labjegyzetek maradnak 1-es sorkozzel!
\usepackage[hang,skip=0pt]{caption}
\usepackage{url}								%For URL-s
\usepackage[headheight=56pt]{geometry}			%For the heading gap


%MATHEMATICAL EXPRESSIONS
\usepackage{bm}									%Bolding
\usepackage[scaled=0.85]{beramono}				%for bolding

%TABLES, TABULATING, DRAWING
\usepackage{array}
\usepackage{tabu}
\usepackage{mdframed}
\usepackage{multirow}
\usepackage{tabularx}
\usepackage{tikz}								%For drawing see --
\usepackage{stuki}								%For struktogramms see --
\usepackage{wrapfig}
\usepackage{hyperref}

\usetikzlibrary[shadows]

%--------------------------------------------------------------------------------------
% Main variables
%--------------------------------------------------------------------------------------
\newcommand{\szerzo}{Hajnal Máté}
\newcommand{\palyazok}{Szedmák Bálint \& Hajnal Máté}
\newcommand{\cim}{BME VIK \& ELTE PPK Gólyabál pályázat \\ Programozás beadandó}
\newcommand{\doktipus}{Palyazat}
%--------------------------------------------------------------------------------------
% Page layout setup
%--------------------------------------------------------------------------------------
% we need to redefine the pagestyle plain
% another possibility is to use the body of this command without \fancypagestyle
% and use \pagestyle{fancy} but in that case the special pages
% (like the ToC, the References, and the Chapter pages)remain in plane style

\pagestyle{plain}
%\setlength{\parindent}{0pt} % áttekinthetõbb, angol nyelvû dokumentumokban jellemzõ
%\setlength{\parskip}{8pt plus 3pt minus 3pt} % áttekinthetõbb, angol nyelvû dokumentumokban jellemzõ
\setlength{\parindent}{12pt} % magyar nyelvû dokumentumokban jellemzõ
\setlength{\parskip}{0pt}    % magyar nyelvû dokumentumokban jellemzõ

\marginsize{35mm}{25mm}{15mm}{15mm} % anysize package
\setcounter{secnumdepth}{0}
\sectionfont{\large\upshape\bfseries}
\setcounter{secnumdepth}{2}
\singlespacing
\frenchspacing

%--------------------------------------------------------------------------------------
% Set up listings
%--------------------------------------------------------------------------------------
\lstset{
	basicstyle=\scriptsize\ttfamily, % print whole listing small
	keywordstyle=\color{black}\bfseries\underbar, % underlined bold black keywords
	identifierstyle=, 					% nothing happens
	commentstyle=\color{white}, % white comments
	stringstyle=\scriptsize\sffamily, 			% typewriter type for strings
	showstringspaces=false,     % no special string spaces
	aboveskip=3pt,
	belowskip=3pt,
	columns=fixed,
	backgroundcolor=\color{lightgray},
} 		
\def\lstlistingname{lista}	

%--------------------------------------------------------------------------------------
%	Some new commands and declarations
%--------------------------------------------------------------------------------------
\newcommand{\code}[1]{{\upshape\ttfamily\scriptsize\indent #1}}

% define references
\newcommand{\figref}[1]{\ref{fig:#1}.}
\renewcommand{\eqref}[1]{(\ref{eq:#1})}
\newcommand{\sectref}[1]{\ref{sect:#1}}
\newcommand{\tabref}[1]{\ref{tab:#1}.}


\author{\szerzo}
\title{\title}
\includeonly{
	titlepage,%
	chapter1,%
	chapter2,%
	chapter3, %
	chapter4, %
	acknowledgement,%
}
%--------------------------------------------------------------------------------------
%	Setup captions
%--------------------------------------------------------------------------------------
\captionsetup[figure]{
%labelsep=none,
%font={footnotesize,it},
%justification=justified,
width=.75\textwidth,
aboveskip=10pt}

\renewcommand{\captionlabelfont}{\small\bf}
\renewcommand{\captionfont}{\footnotesize\it}


%--------------------------------------------------------------------------------------
%	Setup hyperref package
%--------------------------------------------------------------------------------------
\hypersetup{
    unicode=true,              % non-Latin characters in Acrobat’s bookmarks
    pdftitle={\cim},           % title
    pdfauthor={\szerzo},       % author
    pdfsubject={\doktipus},    % subject of the document
    pdfcreator={\szerzo},      % creator of the document
    pdfproducer={Producer},    % producer of the document
    pdfkeywords={keywords},    % list of keywords
    pdfnewwindow=true,         % links in new window
    colorlinks=true,           % false: boxed links; true: colored links
    linkcolor=black,           % color of internal links
    citecolor=black,           % color of links to bibliography
    filecolor=black,           % color of file links
    urlcolor=black             % color of external links
}

%--------------------------------------------------------------------------------------
% Table of contents and the main text
%--------------------------------------------------------------------------------------
\begin{document}
\onehalfspacing
%--------------------------------------------------------------------------------------
%	The title page
%--------------------------------------------------------------------------------------
\begin{titlepage}
\begin{center}
{\huge \bfseries \cim}\\[0.8cm]
\includegraphics[width=140mm,keepaspectratio]{figures/eiffel_tower.png}\\
\vfill
{\LARGE \bfseries \palyazok}\\[0.8cm]
{\large \today}
\end{center}
\end{titlepage}



\tableofcontents\vfill
%----------------------------------------------------------------------------
\chapter{C-RAN koncepció}\label{sect:CranConcept}
A Telekommunikációs ipar az elmúlt években egy robbanásszerű fejlődésen ment keresztül az egyre növekvő forgalomigénynek köszönhetően.
Ez nem csak drasztikusan megnövelte a TCO-t (total cost of ownership, egy megközelítés mely segíti a vevőket és eladókat az egyes termékek árának meghatározására, az összes direkt és indirekt költséget beleszámolva), hanem egyre nehezebbé tette a fenntarthatóságát és az együttműködését a különböző hálózatoknak 2G, 3G, 4G és már beszélhetünk az 5G-ről is.\cite{RecentCRANProg}
A legújabb technológiák, mint a CoMP (Coordinated Multi-Point) ezeket a feladatokat igyekeznek maximálisan ellátni, ám a leghatékonyabb algoritmusaik (pl: JT-Joint Transmission) is kevesek, hogy a hagyományos X2 interface-el elérjék az LTE architektúra maximum teljesítőképességét, mely leginkább a nagy késési időknek és az alacsony sávszélességnek köszönhető. \cite{CoMP}
Végezetül a RAN (Radio Access Network)-ok nagy energia-fogyasztása is megemlíthető.
Ebben a fejezetben az architektúra főbb tulajdonságait és szerepét mutatjuk majd be.
%----------------------------------------------------------------------------
\section{Mi is az a C-RAN?}

\hspace{2mm}
\begin{wrapfigure}{r}{0.4\textwidth}
\captionsetup{format=plain}
\includegraphics[width=0.4\textwidth, keepaspectratio]{figures/power_consumption.png}
\caption{A bázis állomások energia-fogyasztása a 72\%-a az össz energiafogyasztásnak. (China Mobile survey on commercial networks)}
\label{fig:power_consumption}
\vspace{-10pt}
\end{wrapfigure}
A C-RAN egy új RAN architektúra, melyet a China Mobile Research Institute mutattott be 2009-ben. Az akronímát, a tiszta, központosított, együttműködő felhőalapú RAN technológiára (Clean, centralized, collaborative, cloud) használják.\cite{RecentCRANProg}
A megalkotását a megnövekedett forgalomigények és a bázis állomások nagy energia-fogyasztása implikálta. Előbbi évről évre gyakorlatilag megkétszereződik, 2017-ben már a 20 ExaByte-ot is elérheti a Cisco adatai szerint, \figref{traffic}-es ábra. Utóbbiba pedig a China Mobile általános hálózati kutatása ad betekintést, a \figref{power_consumption}-as ábra alapján 72\%-át teszi ki az össz energia-fogyasztásnak.
A C-RAN rendszerek központosítják az alap feldolgozó erőforrásokat úgy, hogy azok egy egységes erőforráskészletet alkossanak, amiből dinamikusan az igényeknek megfelelően lehessen allokálni. Ezzel a technikával rengeteg előnyre tesz szert a hagyományos bázisállomás architektúrával szemben, mint például a megnövekedett erőforrás-használat, a kisebb energia-felhasználás, kisebb interferencia és a CoMP implementációk jobb támogatása. \cite{CoMP}
Fontos kiemelni azt, hogy a C-RAN nem csak egy könnyen alkalmazható technológia létező wireless hálózatok kiterjesztésére, hanem egy kulcsfontosságú elem az 5G-s hálózatok kialakítására.
Ez a technológia könnyen illeszthető megannyi 5G-s technológiához, mint a Large Scale Antenna Systems (LSAS)-hez vagy az ulta-sűrű full-duplex hálózatokhoz.\cite{TechOverview}

\begin{figure}[!ht]
\centering
\includegraphics[width=\textwidth, keepaspectratio]{figures/traffic.png}
\caption{A mobil forgalomigények alakulása (Cisco VNI Mobile 2011)} 
\label{fig:traffic}
\end{figure} 

%----------------------------------------------------------------------------
\section{A C-RAN koncepciója}
\hspace{2mm} \indent Mint említettük, a C-RAN központosítja az alap feldolgozó erőforrásokat, hogy egy egységes erőforráskészletet alkossanak, amiből dinamikusan az igényeknek megfelelően lehet allokálni. Ennek alapkomponensei az elosztott BS-ek (base station-ok, bázis állomások). \cite{RecentCRANProg}
\begin{figure}[!ht]
\centering
\includegraphics[width=\textwidth, keepaspectratio]{figures/cran_arch.png}
\caption{A C-RAN architektúra.\cite{RecentCRANProg}} 
\label{fig:cran_arch}
\end{figure} 

A \figref{cran_arch}-es ábrán szemléltetjük az architektúra három fő elemét.
\begin{itemize}
\item Base Band Unit (BBU) pool: Ez gyakorlatilag az erőforrás-készletünk, mely centralizáltan áll több időben változó mennyiségű soft BBU node-ból. A soft BBU nem más, mint egy hagyományos BBU aminek az erőforrásait és kapacitásit dinamikusan allokáljuk és újrakonfiguráljuk a valós-idejű követelményeknek megfelelve.
\item Remote Radio Unit networks: Ezek hagyomás RRU-k, a koncepció ezeken nem változtat, ők biztosítják az alap wireless lefedettséget
\item Transport networks: Ez az összekötő hálózat a BBU-k és az RRU-k között. Pontosítva egy készleten lévő BBU-t ez köt össze egy megadott RRU-val. Az alkalmazástól függően ez az összekötés teljesen más lehet, példákban láthatunk közvetlen vezetékes pl: sötétszálas kapcsolatot, mikrohullámú szállítást vagy üvegszálas kapcsolatot is.
\end{itemize}
%----------------------------------------------------------------------------
\section{Milyen előnyökkel jár a C-RAN?}
\hspace{2mm} \indent Elsőre csak egy központosított BBU hálózatnak tűnik a C-RAN architektúra, azonban a központosítás csak az első lépés a teljes kép megértéséhez. \cite{RecentCRANProg}
\begin{itemize}
\item BBU központosítás: az első alap feature a C-RAN hálózatokban.
\item Magas technológiai támogatás: A C-RAN pool-jához előkövetelmény egy minden-mindennel összekötött elválasztó-hálózat magas sávszélességgel és alacsony késési időkkel. Az elválasztó mechanizmus érzékeli az összeköttetéseket különböző számoló egységek között és megengedi a leghatékonyabb információáramlást közöttük. Ezek által a rendszer teljesítőképessége megnövekedhet.
\item Erőforrás virtualizáció, felhősítés: A hagyomásos rendszerek számítási kapacitásai limitálva vannak az egyes BBU-k erőforrásaiban és így az egyes node-ok az erőforrásaikat nem tudják megosztani egymással. Esetünkben viszont ezek a node-ok aggregálva vannak egy pool szinten és rugalmasan a kovetelmények alapján lehet őket allokálni.
\item "Soft" BBU: A hagyományos vezetéknélküli felszerelések a tulajdonosok saját platformjai lettek fejlesztve és kizárólag a saját "hard" fix tulajdonságokra tervezték őket, a legmagasabb adatforgalom elviselésére.
Azonban az erőforrások felhősítésével a BBU egy lágyabb "soft" erőforrás lett, melyet dinamikusan újra-konfigurálhatunk és igazíthatunk.
\item A szolgáltatások telepítésének elősegítése a végpontokon: A C-RAN hálózatok nagyobb területket fednek le és több felhasználót látnak el, mint a már megszokott bázis állomások. Ezáltal az egyes szolgáltatások mozgatása, vagy új szolgáltatások telepítése a RAN oldalon egyszerűbbé vált, így egyrészt fokozható a User Experience (felhasználói élmény), másrészről kevésbé terheli a gerinchálózatot.
\end{itemize}
%----------------------------------------------------------------------------
\section{C-RAN Kihívásai}
\hspace{2mm}  \indent Az első probléma, amibe ütközünk az összeköttetések, optikai kábelek mértéktelen szükségessége, melyből mint erőforrás nem áll annyi rendelkezésünkre. A front-haul-t úgy definiálták, mint a BBU-t és az RRU-t összekötő hálózat, a tipikus front-haul protokollok a Common Public Radio Interface (CPRI) és az Open Base-station Architecture Initiative (OBSAI). Az üvegszálhasználat csökkentésére rengeteg kompressziós technikát kitaláltak, de megemlíthetjük az SFDB-t (Single Fiber Bi-direction) is mely megengedi az egy szálon történő szimultán UL és DL, fel és letöltést. Egy másik ötlet a WDM (Wavelength-division multiplexing), mely a hullámfelosztással rengeteg vivőjelet enged meg a szál felosztásával, bár ezzel a zajt és a késések is növelve.
\vspace{-20pt}
\begin{wrapfigure}{r}{0.5\textwidth}
\captionsetup{format=plain}
\includegraphics[width=0.5\textwidth, keepaspectratio]{figures/virtualisation.png}
\caption{RAN virtualizáció}
\label{fig:virtualisation}
\vspace{-20pt}
\end{wrapfigure}

\indent A második nagy kihívás az erőforrások felhőszintű használata, virtualizálása. Mivel eddig mindenki a saját magának megfelelő BBU-t használta, így ki kell fejleszteni egy új BBU entitást, mely mindenkinek kedvező és azon a virtualizációs technológiákon alapul, amelyek a modern Adatközpontokban már jelen vannak. Egy egész jól illeszkedő megoldás lehet a Network Function Virtualization (NFC), ami rengeteg hálózat típust konszolidál ipari nagykapacitású szerverekbe, switchekbe és kapacitásba, amik már elhelyezkedhetnek adatközpontokban, hálózati egységekben vagy akár a usereknél maguknál. \cite{NFV}
A virtualizáció megvalósítását mutatja a \figref{virtualisation}-es ábra. A bázis erőforrás pool több standard IT szerverből áll, és minden szerverhez tartozik egy dedikált hardver gyorsító, hogy a számolásigényes feladatokat a fizikai réteg el tudja látni. Ezeknek a gyorsítóknak meg kell felelniük a valósidejű vezetéknélküli jelfeldolgozási követelményeknek, így az L2/L3-as funkciók egy-egy virtuális gépen futnak, virtuális környezetben. A további felhasználói applikációk, mint a Web-Cache ugyancsak futhatnak ebben a virtuális környezetben. Ezek elsőre könnyen megérthetőnek tűnnek, azonban a real-time feldolgozásnak nagyon szigorú követelményei vannak, így nagyon nagy feladatról beszélünk.\cite{RecentCRANProg}

\hspace{2mm}
\begin{figure}[!ht]
\centering
\includegraphics[width=\textwidth, keepaspectratio]{figures/timeline.png}
\caption{A C-RAN legfontosabb fejlődési lépései} 
\label{fig:timeline}
\end{figure} 

\indent A C-RAN-nak képesnek kell lennie minden a modern hálózatok által elfogadott feladatok ellátására, ezeket majd az implementációnál is figyelembe kell vennünk, így érdemes végignézni a C-RAN fejlődéstörténetét a \figref{timeline}-es ábra segítségével.
\begin{itemize}
\item Valamilyen úton-módon a régebbi rendszerekhez kapcsolódnia kell, amellyel végső soron elérhető lenne a teljes hálózat egyszerűsítése az egységes hálózati architektúrának köszönhetően.
\item Képes kell legyen elválasztani a szabályozási és felhasználói szintet, hogy rugalmasan méretezhető teljesítményt biztosítson a RAN különböző funkciói számára.
\item Számos fejlesztési lehetőséggel kell rendelkeznie a hálózati struktúra területén, (pl. széleskörű transzport hálózati megoldások, bázisállomás konfigurációk és felhasználói alkalmazások)
\end{itemize}
Ha az utolsó implementációs pontot is sikerül meglépnünk, akkor elkezdődhet a tömegszerű felhasználás.
%----------------------------------------------------------------------------
\chapter{Kódolási eszközök}\label{sect:CodeTool}
%----------------------------------------------------------------------------

\hspace{2mm} A feladatomhoz fontos volt a megfelelő kódolási eszközök felmérése és megismerése, ebben a fejezetben ezt részletezem. 

\section{HTML, CSS}
\hspace{2mm} Régebben komoly problémát jelentett, hogy a szabványok nem megfelelő megfogalmazásából adódóan a böngészők eltérően implementálták a dolgokat, így rendszeres megjelnésbeli külnbségek merültek fel. Ez persze az évek során nagyon sokat finomodott, míg végül a World Wide Web Consortium (W3C) lefektetett egy úgy látszik megfelelően magalapozott szabványt a HTML5-tel. Ez a szabvány egységesítette az előtte már kétirányba elhúzódó (XHTML és HTML4) szabványokat. A másik nagy előnye az volt ennek a szabványnak, hogy lehetőséget biztosított a modern kliens oldali fejlesztésnek (Canvas) és az oldal strukturáltáságát is javította renget új elem bevezetésével (article, nav stb...).

A HTML alapvetően csak az oldal logikai struktúrájának leírását szolgálja, és ez a tulajdonsága az évek alatt nem is változott. Nem képes a komplexebb megjelenítési feladatok ellátására, csak az egyes felületelemeket tudjuk megadni ún. tagekkel deklaratív módon.

Mivel a HTML-el nem tudjuk érdemben befolyásolni a lap kinézetét megszületett CSS (Cascading Style Sheets). A CSS segítségével gyakorlatilag minden HTML elemet akár külön-külön meg tudunk formálni, olyanná amilyenre szeretnénk. A CSS szabványok terén még nagyobb volt a káosz, mint a HTML-eknél, főleg amiatt, hogy gyorsan mentek végbe a változások, szerencsére a CSS szabványait végül a CSS3 egybegyúrta.  

%----------------------------------------------------------------------------
\section{Javascript, Typescript}
\hspace{2mm} A jelenleg legnépszerűbb nyelv a webes alkalmazások készítésére a Javascript. Gyakorlatilag minden böngésző támogatja semmilyen beépülő modulra nincs szükség. A böngészők a javascript belső implementációját igyekeztek egységeséteni a szabványnak megfelelően, továbbá a hatékonyság érdekében rengeteget könnyítettek a fejlesztéseken, egyre jobb memóriakezelést alkalmaztak és a böngészőmotorok fel lettek szerelve a töbszálú futás lehetőségével is így nagyobb alkalmazások is könnyen tudnak rajtuk futni.

Magát a nyelvet 1997-ben ECMAScript néven szabványosították először. Az objektumorientált nyelvek családjába tartozik, azon belül is a prototipusos nyelvek közé. Ez a gyakorlatban annyit tesz, hogy nem rendes osztályok vannak a nyelvben, amiket aztán példányosítunk, hanem minden objektum egy prototipus objektum "lemásolásával" jön létre. Ez a nyelv és az egyes objektumok nagyon szabad kezelését teszi lehetővé, gyakorlatilag nyílt objektumaink vannak, amiket szabadon felruházhatunk új attribútumokkal metódusokkal a fejlesztés dinamikus és gyors lehet.

\begin{figure}[!ht]
\centering
\includegraphics[width=0.5\textwidth, keepaspectratio]{figures/javascript.png}
\caption{Egy Javascript objektum} 
\label{fig:JS}
\end{figure} 

A \figref{JS}-es ábra mutat egy példát egy ilyen objektum létrehozására. Fontos megemlíteni, hogy a JavaScript változói dinamikusak, azaz csak futási időben kapnak értéket, és típusuk is csak ekkor derül ki, sőt akár változhat is. Egy további érdekessége a nyelvnek, hogy minden objektumként viselkedik még a függvények is, így értelmezve van mindennél a \textit{this} kulcsszó.

A fenti prototípusosság azonban nagy hátrányt is jelenthet a fejlesztőknek, ugyanis semmilyen megkötés vagy követelés nincsen így arra, hogy az egy objektum milyen tulajdonságokkal rendelkezzen futási időben. Ez nagyon sok kényelmetlenséget okoz a hívó félnek és hívottnak is, ugyanis fel kell készülni, hogy az adott metódus bármilyen paramétert megkaphat, nem köthetjük ki a típusát. Ennek orvoslására rengeteg megoldás született, melyek más-más módokon segítik a fejlesztést. Én ezek közül a TypeScriptet választottam mely egy a Microsoft által fejlesztett scriptnyelv. \cite{Typescript}
A fejleszése során előre készültek az ECMAScript2016-os szabványra, így az átállás nem lesz nehéz. Ez a nyelv gyakorlatilag egy plusz réteget biztosít a Javascript főlé, melyben garantálva van a típusosság. A legjobb tulajdonsága, hogy plain JavaScriptre fordul, azaz egy fordító egy az egyben JavaScript kódot állít elő a TypeScript kódunkból, gyakorlatilag semmilyen böngésző támogatásra nincs szükségünk. 

\begin{figure}[!ht]
\centering
\includegraphics[width=\textwidth, keepaspectratio]{figures/typescript1.png}
\caption{Typescript installáció} 
\label{fig:TS1}
\end{figure} 

A TypeScript fordításához szükség van a NodeJS, npm telepítésére. Ezen kívül a fejlesztő környezet is kérdéses lehet, ugyanis egyre több jó szövegszerkesztő van amiben van TypeScript támogatás. 
A legjobbak talán a Visual Studio Code, a Webstorm, az Atom, az Eclipse és a NetBeans, de van már vimben és Sublime Text-ben is támogatás. A telepítések automatizálására írtam egy shell scriptet (\figref{TS1}), a fejlesztőkörnyezetnek pedig a Sublime Text-et választottam (\figref{TS2}, ennek az editornak van külön package controllerje, amit a script szintén felrak ezután már csak a TypeScript packaget kell letölteni és kész is a támogatás. 

\begin{figure}[!ht]
\centering
\includegraphics[width=\textwidth, keepaspectratio]{figures/typescript2.png}
\caption{Typescript osztály és interfész Sublime Text-ben} 
\label{fig:TS2}
\end{figure} 

A \figref{TS2}-es ábrán látható, egy egyszerű osztály és interfész. Az egyes osztályokat a \textit{module} kulcsszóval lehet csoportokba rendezni, ami hasonló a C++ namespace-eihez. Ha az osztály elé odarakjuk \textit{export} kulcsszót, akkor az az osztály kívülről is elérhetővé válik. Ezeken kívül már hasonló a szintaktika, kell konstruktor és lehet public vagy private változókat megadni, annyi talán a különbség, hogy a változó neve után írjuk a típust kettősponttal elválasztva. Az interfaceken keresztül metódusok érhetők el. A fordítás során keletkezik pár segéd fájl, ilyen pl. a map fájl mely a debuggernek segít a JavaScript TypeScript kódrészletek összepárosításához. 

Meg kell említeni, hogy a nyelv rengeteg újszerű nyelvi elemet tartalmaz. Ilyen például az async/await mely az aszinkron műveletek kezelésében segít, az enum felsorolás típus, az import kulcsszó más fájlokra hivatkozáshoz de ide sorolhatjuk az any kulcsszót is, amivel dinamikusosan típusos változót készíthetünk Typescriptben is.

Ennek a nyelvnek a legnagyobb előnye, hogy a fentiek miatt jól struktúrálható kódot készíthetünk.
%----------------------------------------------------------------------------
\section{Canvas, KonvaJS}

Az alkalmazáson belül a diagram megjelenítésére több lehetőségem volt. Meg lehetett volna oldani csak HTML-el azonban ez a megoldás nem támogatja a különleges alakzatok (nem téglalap) megjelenítését, így ezt hamar elvetettem. Két lehetséges módszer maradt az SVG és a Canvas. Mindkettő megfelelt a feladat elvárásainak, azonban a Canvas mellett döntöttem, mivel ez egy újabb fejlődőbb technika több lehetőséggel. A Canvas pixelpontos rajzolást tesz lehetővé és bármilyen alakzatot könnyedén meg lehet vele rajzolni. A nagy hátránya az, hogyha megrajzoltunk egy alakzatot, akkor arra utána nem lehet már hivatkozni, az alakzatok között semmilyen kommunikáció nincs, nekünk kell elvégezni a nyilvántartást

Utánajárva a problémának az alakzatok közötti eseménykezelést a KonvaJS függvénykönyvtár segítségével oldottam meg. \cite{KonvaJS} Több lehetőségem is volt, ez a függvénykönyvtár a már elavult KineticJS keretrendszerből fejlődött ki, és egy fél éve a KineticJS keretrendszer fejlesztője is ezt jelölte meg alternatívaként, azóta kijött 2016 március 30-án a ConcreteJS keretrendszer is, ám ez még elég fiatal, sok hasznos funkciója van, de még a KonvaJS melett maradtam.

\begin{figure}[!ht]
\centering
\includegraphics[width=0.4\textwidth, keepaspectratio]{figures/konva.png}
\caption{Egy egyszerű kör rajzolása KonvaJS segítségével} 
\label{fig:Konva}
\end{figure} 

A függvénykönyvtár segítségével az egyes alakzatok csoportosíthatók. A \figref{Konva}-es ábra mutatja egy egyszerű kör rajzolását, egy sima API-n keresztül létre lehet hozni az alakzatot. Nagy segítség még a fejlesztéshez, hogy készült hozzá Typescript interfész is leírással.
%----------------------------------------------------------------------------
\chapter{Implementáció}\label{sect:Implement}
%----------------------------------------------------------------------------
\section{Felépítés}

%----------------------------------------------------------------------------
\section{WorkingCalendar}

%----------------------------------------------------------------------------
\section{Schedulers}

%----------------------------------------------------------------------------
\section{Resources}

%----------------------------------------------------------------------------
\chapter{Elosztott rendszerek}\label{sect:distributed}
%----------------------------------------------------------------------------
\hspace{2mm} A legutolsó architektúra amit megnézünk az a legspekulatívabb is egyben.
Tegyük fel, hogy van egy hálózatunk kvantumszámítógépekből, melynek összeköttetései, linkjei képesek qubitek átvitelére.
Ebben a fejezetben felfedezzük, mint is tudnánk elérni egy ilyen rendszerrel.
%----------------------------------------------------------------------------
\section{A szupersűrű kódolás előrevetítése}
\hspace{2mm} Rengeteg hálózati terhelés burstös, kis ideig nagymennyiségű.
Például egy sima böngésző az ideje nagy részében nem hajt végre semmilyen kommunikációt, de minden oldal betöltéskor megabájt nagyságrendű adatot probál letölteni olyan gyorsan, ahogyan lehetséges.

\indent Egy technika, annak érdekében hogy a hálózat kevésbé legyen ilyen burstös, hogy összeköttetés prefetch-elést alkalmazunk, azaz megpróbálja browser megmondani, hogy mi lesz a következő tartalom amit a felhasználó le akar majd tölteni és előre leszedi a háttérben azt, még mielőtt a user rákattintana.
Egy kvantumjelenség az ún. \textit{szupersűrű kódolás} lehetővé teszi a kvantum kliensnek, hogy letöltse, prefetchelje az adatot a kvantumszerverére, úgy hogy a kliensnek még ötlete sincs róla, hogy milyen adatra lesz szüksége a jövőben.

\textbf{Tény 3} (szupersűrű kódolás) Ha egy kliens és egy szerver egy adott összefonódott qubiten osztozik (előre megosztott qubit), akkor a szerver elküldhet két klasszikus bitből álló információt a kliensnek egy szimpla qubit elküldésével.

\indent Ahhoz, hogy implementáljuk a prefetchelést a szupersűrű kódolást használva a szervernek folyamatosan kell összefonódott qubiteket gyártania, melyekből egyet elküld a kliensnek.
Amikor a klens le szeretne tölteni valamilyen adat a gépére, akkor használhatja az összefonódott qubitjét, amin a szerverrel osztoznak, hogy átküldjön adatot \textit{kétszer} olyan nagy bitszámmal, mint amekkora a hálózati összeköttetés kapacitása.
A szupersűrű kódolás megnövelheti a hálózat teljesítményét még akkor is amikor konstans sebességgel tud csak adatot letölteni a szerverről.
Mondjuk azt, hogy a kliens és a szerver egy olyan hálüzati kábellel van összekötve, amelyen az átvitel sebesség 100 Mbps mindkét irányban.
Ezt a szupersűrű kódolást használva, a klens a szerverről \textit{200Mbps}-el tudna adatot letölteni.
Ahhoz, hogy ez sikerüljön összefonódott qubiteket kell küldenie a kliensnek a szerver felé 100Mbps-es sebességel, a szerver pedig visszaküldeni a kliensnek kódolt qubiteket 100Mbps-es sebességgel.
A kliens így a csatornából 200Mbps-es sebességgel tudna információt szerezni, ezzel egy 100Mbps-es kétirányú kommunikációt 200Mbps-es iránymentes kommunikációvá változtatva.
 

%----------------------------------------------------------------------------
\section{Távoli keresés}
\hspace{2mm} Számítógépek kvantum linkkekkel összekötve futtathatnak Grover algoritmust a hálózaton.
Csak, hogy lássunk egy lehetséges alkalmazását ennek az ötletnek: Mondjuk azt, hogy egy kliens szeretne keresni egy speciális elemet egy nagy adattartományon ami $N$ elemből áll $(x_i,...,x_N)$ egy adott szerveren tárolva (pl. lehet szó akár terabájtnyi videó adatról.)
Ebben az esetben lehetne a kliensnek egy osztályozója: $f:\{0,1\}\rightarrow \{0,1\}$ és szeretne egy indexet találni a szervern $i^*$-ot amelyre igaz, hogy $f(x_{i^*})=1$.
A kliens nem szeretné letölteni az egész adatállományt a szerverről és nem is szeretné feltölteni a szerverére az $f$ osztályozóját, lehet hogy ez az osztályozó túl nagy, hogy elküldje vagy van benne valamilyen titkos input, algoritmus.

\indent Egy ilyen esetben tudna akkor a szerveren futtatni egy Grover algoritmust a kliens a hálózaton keresztül, hogy találjon egyező értéket $x_{i^*}$, miközben a szerverrel csak alig $\Theta(\sqrt{N})$ qubitet kellett cserélnie az eredeti klasszikus $\Theta(N)$ bit helyett.
Persze a számolást nem tudjuk megspórolni a kliens és a szerver oldalon, a kliensnek szükséges az osztályozóját kb. $\sqrt{N}$-el lefuttatnia és szervernek is szükséges, hogy lefuttasson kb. $\sqrt{N}$-szer lefuttatnia egy qRAM lekérdezést.
Mindenesetre a hálózati szállításból nyert költség lehet, hogy arányaiban megérték ezt a számolási többletet.
%----------------------------------------------------------------------------
\chapter*{Köszönetnyilvánítás}\addcontentsline{toc}{chapter}{Köszönetnyilvánítás}
%----------------------------------------------------------------------------

Köszönöm szépen a tanár úrnak, hogy megengedte pótleadnom a házi feladatot. Elnézést még egyszer a kellemetlenségért.

%\listoffigures\addcontentsline{toc}{chapter}{Ábrák jegyzéke}
%\listoftables\addcontentsline{toc}{chapter}{Táblázatok jegyzéke}

\bibliography{mybib}
\addcontentsline{toc}{chapter}{Irodalomjegyzék}
\bibliographystyle{plain}


\label{page:last}
\end{document}
