%----------------------------------------------------------------------------
\chapter{Helyszín}\label{sect:Place}
%----------------------------------------------------------------------------
\hspace{2mm} Pályázatunk egyik sarokköve, hogy szeretnénk az idei gólyabált egy új helyszínen megrendezni, méghozzá az \textbf{ELTE Lágymányosi Campus Északi tömb}-jében. Tisztában vagyunk vele, hogy révén ez egy új kezdeményezés, így mind a kivitelezés mind a döntésünk indokoltsága több kérdést vet fel. Ezeket a kérdéseket a következő alfejezetekben igyekszünk megválaszolni.\\
\indent Az új helyszín gondolata már az elmúlt években is felmerült és leginkább anyagi okai vannak/voltak annak, hogy a Dürer rendezvényházban (Dürlin) tartottuk a Gólyabálunkat. Ez a lehetőség tudomásunk szerint idén még fenn áll, így ha bármi probléma merül fel az Északi tömbbel a Dürlin, mint mentő opció tökéletes lehetőség. Ettől az opciótól nem is szeretnénk elzárkózni. Erre a helyszínre már az elmúlt években kialakult egy majdnem optimális elrendezés, ettől ebben az esetben nagymértékben nem is térnénk el.\\
\indent Mivel a Dürlin-t jövőre felújítják, így mindenképpen új helyszínt kell majd keresnünk. Úgy gondoltuk, hogy a legjobb megoldás a probléma elé menni. Tavaly Barbiék nagyon sok helyszínt megnéztek, melyek listáját el is kértük tőlük. Szerintünk ezeknek a köröknek az újra járása nem sok eredményhez vezetne, ez az Északi tömb nyújtja az általunk vélt legjobb megoldást.
\begin{wrapfigure}[10]{l}{0.40\textwidth} 
\begin{center}
\includegraphics[width=0.35\textwidth, keepaspectratio]{figures/moulin1.jpg}
\end{center}
\caption{Moulin Rouge} 
\label{fig:Moulin}
\end{wrapfigure}

%----------------------------------------------------------------------------
\section{Eredet}
%----------------------------------------------------------------------------
\indent Az ötlet, hogy az Északi tömbben rendezzük a bálunkat, az idei SOTE-s Karnevál nyomán merült fel bennünk. Az IÖCS által szervezett Nemzetközi Semmelweis Karnevál egy körülbelül 1900 fős rendezvény melyet minden évben megrendeznek, idén április 2-án került rá sor. A Karnevált rendszerint a NET-ben (Nagyváradi Elméleti Tömb) szokták tartani, ám idén felújítások miatt erre nem volt lehetőségük, így jutottak el az Északi tömbhöz, mint helyszín.\\
\indent	Megkerestük a Karnevál egyik főszervezőjét, Klément Emesét (Mesit) a részletekkel kapcsolatban, aki rengeteg kérdésünkre válaszolt és egy nagyon pozitív feedbacket adott a helyről. Mesitől elkérve a hely kontaktját, jutottunk el Garab Emilhez (egarab@gmail.com, 20/925-7422), aki a Lágymányosi épületek rendezvényekkel kapcsolatos felelős személye. Emillel közös beszélgetéseink során igyekeztünk minden részletet megtudni, melyek a megvalósításhoz kellenek, ezekre Ő készségesen válaszolt és még a legutolsó kérdéseinket is tisztázta egy az Északi tömbben való tárlatvezetés alkalmával.

%----------------------------------------------------------------------------
\section{Új helyszín pro-kontra}
%----------------------------------------------------------------------------
\subsection{Ár}
\hspace{2mm} Az első nagy kérdés számunkra a helyszín ára volt. A SOTE-sok összesen bruttó 1,53milllió forintért bérelték a helyszínt. A jelenlegi lovagrendi gazdaságist Barbit megkérdezve, a Dürlin 900 ezer forintba került.  Első látásra mi is elszomorodtunk, de kiderült, hogy a bérlés sok részre diverzifikálható. A Karnevál majd 700 fővel nagyobb rendezvény, mint a miénk, így mi nem igénylünk akkora teret mint ők.\\
\indent A helyszínbérlés náluk négy összetevőből állt, 582 ezer forintért bérelték a Gömb Aulát, 218 ezer forintért a Gömb Aulához tartozó karzatot, 400 ezer forintért a Harmónia termet és 4 nagy tantermet darabonként 72, valamint 5 kicsit darabonként 8 ezer/14 ezer forintért. A helyszínből, ami mindenképpen kell nekünk az a Gömb Aula, a felette lévő karzat és legalább kettő kis terem, így árban kicsit több, mint 800 ezer forintét már tudunk helyszín-ügyileg itt Gólyabált rendezni.  A helyszínt bejárva viszont az optimális (számunkra szerintünk legideálisabb) elrendezést tekintve, 3 nagy tanterem és 3 kicsi tanterem kell (a Harmónia terem semmiképp sem), így a végösszeg kicsit több, mint 1,05 millió forintra jön ki, amely 150 ezer forinttal jelent többet, mint a Dürlin. Mivel a helyszín rengeteg előnnyel jár, ez a plusz költség rengeteg más helyen megtérül (tranzit, gombák, sátrak bérlése jelentős kiesést jelent, plusz Gólyabál busz sem kell), így ez a plusz költség szerintünk belefér. Természetesen hogyha mégsem fér bele módosítjuk majd a terveinket, mindenesetre pályázatunkban ez alapján terveztünk és Emiltől is ezek alapján már kértünk egy előzetes árajánlatot.

\subsection{Kontra}
\hspace{2mm} A helyszín legnagyobb -- és egyetlen -- ellenérve az, hogy mivel az ELTE-n hétköznap tanítás van, így a bált csak szombati napon tudjuk megrendezni. Az ELTE PPK-n már 4 éve gond, hogy szeretnék a bált csütörtök megtartani, mert náluk “mindenki” haza megy már csütörtök este, így a pénteki időpontokba is ódzkodva mentek bele. Továbbá, nem tudhatjuk hány embert veszítünk el a pénteki helyett szombati időpont miatt (a péntek egyszerűen csak ideálisabb a gólyáknak). \\
\begin{wrapfigure}[12]{r}{0.2\textwidth} 
\begin{center}
\includegraphics[width=0.18\textwidth, keepaspectratio]{figures/eiffel2.jpg}
\end{center}
\caption{Eiffel torony} 
\label{fig:Eiffel}
\end{wrapfigure}
\indent Fontos megemlíteni, hogy a szombati időpont a fentebb említettekkel szemben pozitívumokat is rejt magában. Először is a rendezés során rendszeresen probléma volt, hogy péntek reggel sok embernek még ZH-ja volt, így sok rendező nem tudott bevállalni egy reggeli pakolást vagy csütörtök esti utolsó gyűlést. Ez a probléma egy az egyben megszűnne. Rendezői szempontból ideálisabb a szombati időpont. Ehhez még az is hozzájön, hogy már péntek este elkezdhetjük a helyszín berendezését (ezt Emil kiemelte), így erre bőven elég időnk lenne, nem szükséges a feszített tempó. \\
\indent A szombati időpontnak hála lehetne a Gólyabálnak egy íve. Megkérnénk a ClubCeption-t, hogy csináljanak a Gólyabál előtti péntek este 9-től egy Gólyabál váró bulit, rákészülésképpen és ugyanezen az estén tartanánk egy gyűlést a kisfőnökökkel -- PPK-sok is -- az ENT-ben 8-tól, melyen az utolsó kérdéseket is elsimítanánk. Szombaton megtörténne a Gólyabál és vasárnap Kakasban éjfél kispulton várnánk a (jobb esetben :P) kipihent rendezőket. \\
\indent Úgy érezzük, hogy a PPK-sok a fentebbi érveinket végighallgatva meg fogják érteni a problémánkat és bele fognak egyezni a szombati bál gondolatába.



\subsection{Pro} 
\hspace{2mm} Az új helyszínnek rengeteg előnye van, melyeket a teljesség igénye nélkül szeretnénk felsorolni.
\begin{itemize}
	\item  A Dürlin legnagyobb hátránya az, hogy a táncokat nem lehet jól látni, ezért vetítésekkel oldjuk meg a közvetítést. Ez a probléma teljesen megszűnik, hiszen a Gömb Aulát a karzatról tökéletesen beláthatja az egész közönség.
	\item A Gömb Aula akusztikája világhíres. Hogyha középen dobbant valaki azt az Aula túlsó felében is hallani lehet. Ez hihetetlenül meg tudja emelni a koncertek minőségét és a hangulatot.
	\item A Gömb Aula teteje üvegből van. Ez lenne az első VIK-es Gólyabál, ahol a Holdfénykeringő valóban a holdfényben történik. Az egész Gömb Aula világítását egy helyről lehet kezelni, így a lámpák lekapcsolása egyszerűen kivitelezhető.
	\item Az Északi tömb a rakparton helyezkedik el és mellette található egy móló, ahol ki tud kötni a Gólyahajó, melynek utasai egyből a bálhelyszínre tudnának sétálni. Nem lenne szükséges még villamosozni, mint a Dürlinnél.
	\item A bejárat mellett csigalépcsők visznek le a -1-re, ahol a ruhatárazás tökéletesen megoldható. Itt meg kell említenünk, hogy a helyszín biztosít elegendő fogast, így nem kell a NET-ből kölcsön kérnünk.
	\item A három nagy tanteremből kettő egybe nyitható, melyekben a Borozó és a Kajáspult is kényelmesen elfér. A harmadikban pedig elegendő helyet tudunk biztosítani a Parkettnek.
	\item A tantermekben a padok és asztalok mozgathatók, melyeket oda vihetünk ahova szeretnénk, így nem szükséges a Zé épületből padokat kérnünk, az összes pultot meg tudjuk csinálni iskola asztalokból. 
	\item Mivel nem kell a Zé épületből asztalokat kérnünk egy kis tranzit elég a Gólyabál pakoláshoz, nem kell a teherautó. Ez a Gólyabál költségvetésében is jelentős kiesés. Természetesen plusz sörpadokat még kell majd kérnünk más koliktól pl: Martos, de ez csak egy vagy kettő plusz kis-tranzit kört jelent.
	\item A meghívott körök nyitása a Dürlinhez hasonlósan megoldható a helyszínen. (részletek a következő alfejezetben.)
\end{itemize}

%----------------------------------------------------------------------------
\section{Helyszín tervek}
%----------------------------------------------------------------------------
\hspace{2mm} A helyszínt bejárva, végiggondoltuk az események megvalósítását, melyet igyekeztünk vizuális formában is minél jobban megragadni. Az Északi tömbnek három bejárata van, egy a rakpart felől egy a tudósok körútja felől és egy a Déli tömb felől. Ezek közül a Déli tömb felőli a legszélesebb és kényelmesebb, a többit lezáratnánk. Fontos megemlíteni, hogy ezeket a bejáratokat fizikailag nem lehet becsukni, hisz bármelyik egyetemi tanárnak biztosítanunk kell tudni a bejárást még hajnalban is, így ezekbe rendezőket és biztonsági őröket kell majd állítani, akiknek koordinálását a kapus kisfőnökök fogják végezni. 

\subsection{Beengedés:}
\hspace{2mm} A bejárat előtt lenne a beengedés, itt van egy nagyobb területünk a kordonok lerakásához, és a beengedéshez szükséges infrastruktúra kialakításához. Az egész ELTE kampuszon van Wifi lefedettség, aminek eléréséhez csupán Ceasar azonosítóra van szükség, mellyel minden ELTE-s hallgató rendelkezik (Máténak is van), így ez semmiképp sem jelenthet problémát (Emil azt mondta egy közös, a rendezvényhez tartozó azonosítót is kapnánk).
\indent A főbejárat előtt a beengedés után van egy nagyobb füves rész, melyet engedéllyel használhatunk, és maga az engedély sem drága (Emil azt mondta, hogy nem több 15-20 ezer forintnál). Ennek a füves területnek a nagysága elegendő ahhoz, hogy elhelyezzünk rajta egy szigorúan nem leszúrt akár 6*6 méteres sátrat, melyben a WTF kényelmesen elfér. A területhez az engedélyt és egyébként az egész bérléssel kapcsolatban a szerződést Fehér Ágnessel a kampusz igazgatójának segítségével kötnénk meg. Emil azt mondta, hogyha úgy adódik, akkor szívesen bemutat minket neki. Megemlítenénk még, hogy itt a bejárat előtt van lehetőségünk egy dohányzásra kijelölt terület kialakítására is.

\subsection{Biztonsági őrzés \& tűzvédelem}
\hspace{2mm} A kampuszon a biztonsági őrzésért egy külön cég felel a Dussmann Kft, melynek vezetője Pásztor Zsolt (pasztor.zsolt@dussmann.hu, 20/559-3607). A helyszín ragaszkodik hozzá, hogy valamilyen arányban szerepeljen emberük a rendezvényen, ám nem szükségszerűen kell, hogy ők biztosítsák a teljes biztonsági őrzést. Pontosabban is utánajártunk így a dolognak és számukra elég, ha legalább három biztonsági őr lesz jelen tőlük a rendezvényen. Ebbe úgy érezzük, hogy Orbán Balázsék is bele fognak menni, hiszen az előre láthatólag 11 biztonsági őrből a miénk lesz 8 fő, ami kellemes arány. Zsolt felel az épület tűzvédelmi szabályzatának [3] megfelelő betartásáért is, és felajánlotta nekünk, hogy szívesen csinál egy tűzvédelmi tárlatvezetést a rendezőknek. Ezzel a lehetőséggel szeretnénk is élni, úgy véljük a kapu- és lengő főnököknek semmiképp sem árthat egy ilyen előadás. Természetesen csak számukra lenne ez “kötelező” jellegű más kisfőnököknek nem. 

Készítettünk pár tervezetet a belső elrendezésről, de mivel ezek még folyamatosan változhatnak, így csak később tervezzük minden részletében kidolgozni.
\begin{figure}[t]
\centering
\includegraphics[width=0.6\textwidth]{figures/stamps.jpg}
\caption{Bélyegek}
\label{fig:Stamps}
\end{figure}
