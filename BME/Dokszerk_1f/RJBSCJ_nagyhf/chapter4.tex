%----------------------------------------------------------------------------
\chapter{Programozás házi feladat}\label{sect:ProgHF}
%----------------------------------------------------------------------------

%----------------------------------------------------------------------------
\section{Feladat}
%----------------------------------------------------------------------------

\textit{\indent Egy étteremben a pincérek által felvett rendeléseket egy szekvenciális input fájlban tartják nyilván az ételek neve, azon belül a rendelések időpontja szerint rendezett formában. Feltehetjük, hogy a fájl nem üres. A tárolt adatok: a rendelt étel neve, a rendelés időpontja, rendelt adagok száma, egy adag ára. Melyik étel hozta az étteremnek a legtöbb bevételt (összesített darab*egységár)?}

%----------------------------------------------------------------------------
\section{Specifikáció}
%----------------------------------------------------------------------------
A feladat állapottere többféleképpen felírható
% A feladat állapottere
\begin{flalign*}
	A=(f:Infile(&\mathbb{K}),~cout:Outfile(\mathbb{K}))\\
	A=(f:Infile(&Sor),~cout:Outfile(Sor))\\
	A=(f:Infile(&Rendeles), cout:String)\\
	&Rendeles=\textbf{rec}(nev:String, ido:\mathbb{N}, adag:\mathbb{N}, ar:\mathbb{N})
\end{flalign*}
\indent Számunkra a legideálisabb már egy olyan felsorolón dolgozni, ami rendelés nevű rekordokat dolgozni, melyek a rendelések nevét és hozzájuk tartozó bevételeket tartalmazzák. Erre a felsorolóra kell nekünk egy maximum keresést írni.\\
% Összegzés, összefűzés
A maximumkeresés:
\begin{flalign*}
	&A=(t:Enor(Rendeles),~cout:String,~max:\mathbb{N},~elem:Rendeles)\\
	&\hspace{25mm}Rendeles=\textbf{rec}(nev:String, bevetel:\mathbb{N})\\
	&Ef=(t=t'~\wedge~|t'|>0~\wedge~t.azon\uparrow)\\
	&Uf=((max,~elem)=\max_{e\in t'}{(e.bevetel)}~\wedge~cout=elem.nev)
\end{flalign*}

%----------------------------------------------------------------------------
\section{Absztrakt program}
%----------------------------------------------------------------------------
\begin{table}[htb]
\caption{Absztrakció}
\label{tab:Abstraction}
\begin{center}
\begin{tabular}{|lll|}
	\hline
	\multicolumn{3}{|c|}{\textbf{Maximum~keresés}}\\
	\hline
	t & $\sim$ & \textit{Rendelések bevételeivel való felsorolója}\\
	+,0 & $\sim$~ & \textit{+,0}\\
	f(e) & $\sim$ & \textit{e.bevetel}\\
	\hline
\end{tabular}
\end{center}
\end{table}

\noindent\hfill
	\begin{stuki}[12cm]
		\stm{t.First()}
		\stm{max,~elem:=t.Current().bevetel,~t.Current()}
		\stm{t.Next()}
		\begin{WHILE}{4}{\stm{\lnot t.End()}}
			\begin{IF}[70]{2}{\stm{t.Current().bevetel>max}}
				\stm{max,~elem:=t.Current().bevetel,~t.Current()}
			\ELSE
				\stm{SKIP}
			\end{IF}
			\stm{t.Next()}
		\end{WHILE}
		\stm{cout:=elem.nev}
	\end{stuki}
\vspace{5mm}
\textbf{Rendelések bevételeivel való felsorolója:}\\
	\begin{table}[htb]
	\caption{Felsoroló}
	\label{tab:Felsorolo}
	\begin{center}
	\begin{tabular}{|l|l|}
		\hline
		\textit{enor(Rendeles)}& \textit{First(),Next(),Current(),End()}\\
		\hline
		\textit{f:Infile(nev:String,ido:$\mathbb{N}$,adag:$\mathbb{N}$,ar:$\mathbb{N}$) }& \textit{First()$\sim$ st, $akt_f$, f:read;Next()}\\
		\textit{st:Status} & \textit{Next()$\sim$ lsd.külön}\\
		\textit{elso:String} & \textit{Current()$\sim$ akt}\\
		\textit{akt:Rendeles} & \textit{End() $\sim$ st=abnorm}\\
		\textit{$akt_f$:rec(nev:String,ido:$\mathbb{N}$,adag:$\mathbb{N}$,ar:$\mathbb{N}$)} & \\
		\hline
	\end{tabular}
	\end{center}
	\end{table}
\indent A First művelettel beolvastunk a szekvenciális input fájlunkból egy rendelés elemet, melyet az $akt_f$ rekordban tárolunk. A beolvasás sikerességét a státusz jelzés jelzi nekünk.\\
\indent A Next művelet több feladatot, kell ellásson. Addig kell olvasson a szekvenciális input fájlból, amíg a rendelés nevek megegyeznek, vagy véget nem ér a fájl. Ehhez először egy elso nevű változóban eltároljuk az aktuális rendelés nevét, továbbá ehhez hasonlóan egy akt nevű változóban eltároljuk a szükséges rendelés paramétereket, (név és bevétel=adag*ár). Az összegzést, úgy valósítjuk meg, hogy az $akt_f$-be beolvasott input fájl értékek segítségével (az adag*ár értékekkel) az aktuális felsoroló elem $akt$ bevétel nevű változóját növelgetjük. A ciklus akkor marad abba, ha a beolvasás nem sikerült, vagy új rendelésünk van.
\begin{flalign*}	
	&A^{Next}=(f:Infile(nev:String,ido:\mathbb{N},adag:\mathbb{N},ar:\mathbb{N}),elso:String,st:Status,\\
	&\hspace{30mm}act_f:rec(nev:String,ido:\mathbb{N},adag:\mathbb{N},ar:\mathbb{N}),akt:Rendeles)\\
	&Ef^{Next}=(f=f^1~\wedge~st=st^1~\wedge~akt_f=akt_f^1)\\
	&Uf^{Next}=((st=norm\rightarrow elso=akt_f^1.nev\\
	&\hspace{30mm}\wedge~akt.bevetel^2=0)~\wedge~\\
	&(st^2,{akt_f}^2,f^2)=((akt.nev,akt.bevetel)=\sum\limits_{akt_f\in {akt_f}^1,f^1}^{elso=akt_f.nev \wedge st=norm}{(akt_f.nev,} \\
	&\hspace{10mm}akt.bevetel+akt_f.adag*akt_f.ar)\\
\end{flalign*}

\begin{table}[htb]
\caption{Összegzés}
\label{tab:Osszegzes}
\begin{center}
\begin{tabular}{|lll|}
	\hline
	\multicolumn{3}{|c|}{\textbf{Összegzés}}\\
	\hline
	+,0 & $\sim$~ & \textit{+,0}\\
	f(e) & $\sim$ & $akt.bevetel:(akt_f.adag*akt_f.ar)$ \\
	\hline
\end{tabular}
\end{center}
\end{table}

\noindent\hfill
\begin{stuki}[12cm]
	\begin{IF}[70]{2}{\stm{st=norm}}
		\stm{elso,akt.nev,akt.bevetel=\\
		akt_f.nev,0}
	\ELSE
		\stm{SKIP}
	\end{IF}
	\begin{WHILE}{2}{\stm{elso=akt_f.nev \wedge st=norm}}		
		\stm{akt.nev,akt.bevetel:=akt_f.nev,akt.bevetel+akt_f.adag*akt_f.ar}
		\stm{st,akt_f,f:read}
	\end{WHILE}
\end{stuki}


%----------------------------------------------------------------------------
\section{Implementáció}
%----------------------------------------------------------------------------
Program váz: A program több állományból áll: \texttt{etterem.cpp, enor.h, enor.cpp}
\begin{table}[htb]
\caption{Állományok}
\label{tab:Allomanyok}
\begin{center}
\begin{tabular} {|l|l|l|}
	\hline
	\textbf{etterem.cpp} & \textbf{enor.h} & \textbf{enor.cpp} \\
	\hline
	\texttt{int main()} & \texttt{struct} $Akt_f$ & \texttt{Enor::Read()} \\
	& \texttt{struct Rendeles} & \texttt{Enor::Enor()} \\
	& \texttt{Enor()} & \texttt{Enor::Next()} \\
	& \texttt{void First()} & \\
	& \texttt{void Next()} & \\
	& \texttt{Rendeles Current()} & \\
	& \texttt{bool End()} & \\
	\hline
\end{tabular}
\end{center}
\end{table}
A függvények kapcsolódási szerkezete:\\
\begin{center}
	\begin{tikzpicture}
		[grow=right, align=center, root/.style={rectangle, draw=black, thick, drop shadow, fill=white, text width=5em, rounded corners, minimum height=2em}, leaf/.style={rectangle, draw=black, thick, drop shadow, fill=orange!40, text width=5em, rounded corners, minimum height=2em}, level distance=5cm];
		\node [root] (main) {\textbf{main()}}
			child {node [leaf] (current) {\textbf{Current()}}}
			child {node [leaf] (end) {\textbf{End()}}}
			child {node [leaf] (next) {\textbf{Next()}}
				child {node [leaf] (readb) {\textbf{Read()}}}
			}
			child {node [leaf] (first) {\textbf{First()}}
				child {node [leaf] (reada) {\textbf{Read()}}}
			}				
			child {node [leaf] (enor) {\textbf{Enor()}}};
	\end{tikzpicture}\vspace{2mm}\\
\end{center}\newpage
A felsoroló osztálya:
\begin{mdframed}
	\texttt{
	\begin{tabbing}
	\hspace{1cm}\=\textbf{class} Enor\{ \+\\
	    \hspace{1cm}\=\textbf{private:} \+\\
	        \hspace{1cm}\=std::ifstream f;\+\\
	        Status st;\\
	        Rendeles akt;\\
	        $Akt_f$ $akt_f$;\\
	        std::string elso;\\
	        \textbf{void} Read();\\
	\-\\
	    \textbf{public:}\+\\
	        Enor(const std::string \&str);\\
	        \textbf{void} First() \{Read(); Next();\}\\
	        \textbf{void} Next();\\
	        Rendeles Current() \textbf{const} \{ return akt;\}\\
		\textbf{bool} End() \textbf{const} \{ return st==abnorm;\}\-\-\\
	\};\-\\
	\end{tabbing}
	}
\end{mdframed}
\vspace{2mm}
\hspace{2mm} Az osztályon belül az input fájl tartalmát egy streamben tároljuk $(ifstream)$, melyből a beolvasás a $<<$ operátor segítségével történik. Az input fájl neve tetszőleges lehet, esetünkben az \texttt{input.txt} nevet kapta. A Read() függvény tölti fel az operátor segítségével az $akt_f$ struktúra változónkat a megfelelő értékekkel.\\
\indent Az aktuális rendelést az $akt$ változóban tároljuk, melynek típusa szintén egy struktúra, neve Rendeles.

%----------------------------------------------------------------------------
\section{Tesztelési terv}
%----------------------------------------------------------------------------
\hspace{2mm} A megoldás során két programozási tételt alkalmaztunk az összegzését és a maximum keresését is. Ebből az összegzés tétele annyiszor történik meg, ahány különböző nevű rendelés van. A tesztesetek tehát két részre bomlanak, a maximum keresés teszteseteire és az összegzés teszteseteire, előbbit az utóbbi minden permutációjára meg kell nézni.\vspace{2mm}\\
A fekete doboz tesztesetei:
\renewcommand{\labelenumi}{\Alph{enumi}.}
\renewcommand{\labelenumii}{\arabic{enumii}.}
\begin{enumerate}
	\item \textbf{Maximum keresés} tesztesetei:\\
		\textbf{intervallum hossza} szerint
		\begin{enumerate}
			\item Üres állomány 
			\item Egyetlen üres sort tartalmazó állomány
			\item Egyetlen rendelés
			\item Több rendelés
		\end{enumerate}
		\textbf{intervallum elejes és vége} szerint
		\begin{enumerate}
			\setcounter{enumii}{4}
			\item Az intervallum elején van
			\item Az intervallum közepén van 
			\item Az intervallum végén van
		\end{enumerate}
	\item \textbf{Összegzés} tesztesetei
		\begin{enumerate}
			\setcounter{enumii}{7}
			\item Egy elemet kell összegezni
			\item Több elemet kell összegezni
		\end{enumerate}
\end{enumerate}\vspace{2mm}
A megoldó programra épülő (fehér doboz) tesztesek:
\renewcommand{\labelenumi}{\arabic{enumi}.}
\begin{enumerate}
	\item Hibás vagy nem létező állománynév megadása.
	\item Nem megfelelően kivitelezett bemeneti állomány -> nem megfelelő eredmény, de a beolvasások megtörténnek.
\end{enumerate}
Mindkét tesztesetet a feladat szerint kizárhatjuk.


%----------------------------------------------------------------------------
\section{Hivatkozások}
%----------------------------------------------------------------------------
\hspace{2mm} Hogy pályázatunk szép legyen így a \sectref{Visions}. fejezetben elhelyeztünk egy fekete ruhás hölgy képét (\figref{FeketeNo}). Hogy ez az imázs megmaradjon a \sectref{Place}. fejezetben is folytattuk a képek beillesztését (\figref{Moulin}, \figref{Eiffel}, \figref{Stamps}). A \sectref{ProgHF}. fejezeben lévő  maximumkereséseshez és összegzéshez tartozó táblázatok (\tabref{Abstraction}, \tabref{Felsorolo}, \tabref{Osszegzes}, \tabref{Allomanyok}) biztosítják számunkra a megfelelő mennyiséget. 
\begin{align}
\label{eq:Integral}
\int_{-\infty}^\infty \mathrm{e}^{-\alpha x^2} \mathrm{d}x = \sqrt{\frac{\pi}{\alpha}}
\end{align}

A fentebbi \eqref{Integral} egyenlet pedig még egy megfelelő képlettel is szolgál nekünk. 