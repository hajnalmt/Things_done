%Feltöltendő:
%1 darab fájl, lehetőség szerint zip, mely tartalmazza a forrás tex fájlt, valamint a lefordított pdf fájlt.
%A fájl elnevezése legyen a következő: neptun-kód_kishf1.zip

%Feladat:
%Hozz létre egy Report dokumentumot, ami az első oldalon tartalmazza a szerzőt, a mű címét és a készítés dátumát.
%A dokumentum betűmérete 12pt legyen és Times betűtípust használj!
%A mű tartalmazzon tartalomjegyzéket!
%Az oldalszámozás kisbetűs római legyen!
%A létrejövő mű tartalmazzon egy Bevezető és egy Összefoglaló nevű szakaszt!
 %A szakaszok tartalmazzanak alszakaszokat is! Az alszakaszokban írott szöveg legyen másfeles sorközű és gyakorold a megtanult szövegformázásokat (pl. új bekezdés, kiemelés, betűméret)
 %Opcionális gyakorlás: egy széljegyzetet ill. élőfej és élőlábat illessz be a műbe!

\documentclass[12pt,a4paper]{report}	%Report dokumentum
\usepackage[utf8]{inputenc}				%UTF-8-as kódolás
\usepackage{t1enc}						%Furcsa betűk
\usepackage[magyar]{babel}				%Magyar nyelv
\usepackage{mathptmx}					%Times New Roman font
\usepackage{fancyhdr}

\pagestyle{fancy}

\fancyhead{}
\fancyfoot{}
\fancyhead[R]{Szép élőfej}
\fancyfoot[R]{Szép élőláb}

\begin{document}
\title{Első házifeladat}
\author{Hajnal Máté}
\date{\today}

\maketitle

\tableofcontents\vfill
\pagenumbering{Roman}

\chapter{Bevezető}
\section{Irodalmi áttekintés}
Itt jön a szöveg maga\dots

\subsection{Küldetésünk (Mik a céljaink?)}
Célunk a Schönherz Bázisnál, hogy a velünk kapcsolatba került informatikusok és villamosmérnökök karrierjét folyamatosan támogassuk, követve karrierútjukat és a változó munkaerõ piaci igényeket. Minden életszakaszban, akár pályakezdõ, akár sokat látott mérnökrõl legyen szó, tapasztalatunkra és átfogó piac ismereteinkre támaszkodva olyan tanácsokkal tudunk szolgálni, ami segít a válaszút elõtt álló szakembereknek következõ munkahelyük megtalálásában.
\pagebreak
\subsection{Víziónk (Hogyan képzeljük el mindezt?)}
Hisszük, hogy minden megfelelõ szaktudással rendelkezõ mérnök számára létezik „ideális” munkahely. Mi segítünk ennek megtalálásában, kiválasztásában. A Schönherz környékén lassan 20 éve foglalkozunk azzal, hogy fiatal mérnököket „életre” neveljünk. Valós körülmények között próbálhassák ki magukat, sze\-rezhessenek tapasztalatot és bõvítsék tudásukat gyakorlati ismeretekkel. Ez idõ alatt nagyon sok céggel és fiatal mérnökkel kerültünk kapcsolatba, ismerjük a lehetõségeiket és a problémáikat. Így minden rendelkezésünkre áll, ahhoz hogy a megfelelõ jelöltet mutathassuk be ideális munkahelyeket kínáló partnereinknek.

\newpage
\chapter{Összefoglaló}
\section{Irodalmi áttekintés}
Itt jön a szöveg maga\dots

\subsection{Küldetésünk (Mik a céljaink?)}
Célunk a \emph{Schönherz Bázisnál}, hogy a velünk kapcsolatba került informatikusok és villamosmérnökök karrierjét folyamatosan támogassuk, követve karrierútjukat és a változó munkaerõ piaci igényeket. Minden életszakaszban, akár pályakezdõ, akár sokat látott mérnökrõl legyen szó, tapasztalatunkra és \MakeUppercase{átfogó piac} ismereteinkre támaszkodva olyan tanácsokkal tudunk szolgálni, ami segít a válaszút elõtt álló szakembereknek következõ munkahelyük megtalálásában.\marginpar{Tökletes trükk}
\nopagebreak
\subsection{Víziónk (Hogyan képzeljük el mindezt?)}
{\Huge Hisszük}, hogy minden megfelelõ szaktudással rendelkezõ mérnök számára létezik „ideális” munkahely. Mi segítünk ennek megtalálásában, kiválasztásában. A Schönherz környékén lassan 20 éve foglalkozunk azzal, hogy fiatal mérnököket „életre” neveljünk. Valós körülmények között próbálhassák ki magukat, sze\-rezhessenek tapasztalatot és bõvítsék tudásukat gyakorlati ismeretekkel. Ez idõ alatt nagyon sok céggel és fiatal mérnökkel kerültünk kapcsolatba, ismerjük a lehetõségeiket és a problémáikat. Így minden rendelkezésünkre áll, ahhoz hogy a megfelelõ jelöltet mutathassuk be ideális munkahelyeket kínáló partnereinknek.

\end{document}