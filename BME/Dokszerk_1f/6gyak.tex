\documentclass[a4paper]{article}
\usepackage[utf8]{inputenc}
\usepackage{t1enc}
\usepackage[magyar]{babel}
\usepackage{amsmath}
\usepackage{amsthm}
\usepackage{mathtools}

\renewcommand{\proofname}{Bizonyítás}
\newtheorem{tetel}{Tételcím}
\begin{document}
Ha $a$ kisebb, mint $b$, akkor

\[
a^2+b^2=c^2
\]
$R_{ui}=t^{a+b}\cdot r_{e2}$
\begin{equation}
a^2+b^2=c^2
\end{equation}

\[
x \geq 0, \textrm{ ha $x$ nem negatív}
\]

 \begin{equation}
\label{eq:emeletes}
A+b=c
\end{equation}

\section{Első szakaszom}
\subsection{Itt meg az alszakasz}
\begin{equation}
x \geq 0, \textrm{ ha $x$ nem negatív}
\frac{\frac{\pi}{2}}{\frac{1}{2}} and {n \choose k}
\end{equation}

\begin{equation}
\label{eq:emeletes}
\lim_{n \to \infty} \left( \sqrt {
\frac 1 { \frac 1 n + \sqrt {
\frac 1 { \frac 1 n + \sqrt {
\dots }}}}} \right) = ?
\sqrt[3]{2} e \parallel f
\end{equation}

$HCl\overset{\left[\frac{2}{3}\right]}
{\underset{\mathrm{B}}{\rightarrow}}H^+_2+Cl^-$

\[
c^2 = \overbrace{
\underbrace{a^2}_\text{Szöggel szemközti befogó} +
\underbrace{b^2}_\text{Szög melletti befogó}
}^\text{Geometria alaptétele}
\]
\[
c^2 = \underbracket{
\underbrace{a^2}_\text{Szöggel szemközti befogó} +
\underbrace{b^2}_\text{Szög melletti befogó}
}_\text{Geometria alaptétele}
\]

$\begin{pmatrix} a&b\\ c&d \end{pmatrix}$

\begin{multline}\label{címke}
1+8+27+64=\\
=1+3+5+7+{}\\
+9+11+13+{}\\
+15+17+19
\end{multline}

\begin{equation}\label
{cimke}
\begin{split}
100 &=1+8+27+64=\\
&=1+3+5+7+9+{}\\
&\quad+11+13+15+17+19
\end{split}\tag*{A-1}
\end{equation}

\[
\begin{split}
\cos^2\alpha &=1-
\sin^2\alpha=\\
&=1+3+5+7+9+{}\\
&\quad+11+13+15+17+1
9
\end{split}
\]

\begin{eqnarray*}
A&=&B,\\
C&=&D,\\
E&=&F
\end{eqnarray*}
\begin{align*}
A&=B,\\
C&=D,\\
E&=F
\end{align*}

\begin{subequations}
Itt jonnek a felsorolt egyenletek:
\begin{align}
x^2 + y^2 &= 1\\
y &= \sqrt{1 - x^2}.
\end{align}
\end{subequations}

\[
u(x) =
\begin{cases}
\sqrt{x}& \text{Ha } x \geq 0 \\
1 & \text{Ha } x = 0 \\
0 & \text{egyébként}.
\end{cases}
\]

\begin{proof}
A bizonyítás szövege.
\end{proof}
\begin{tetel}\label{xy}
A tétel szövege.
\end{tetel}
\begin{proof}[\Aref{xy}.~tétel bizonyítása]
A bizonyítás szövege.
\end{proof}
\end{document}