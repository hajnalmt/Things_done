\documentclass[a4paper]{article}
\usepackage[utf8]{inputenc}
\usepackage{t1enc}
\usepackage[magyar]{babel}

\sloppy

\begin{document}


\title{Az aranyhalak nemi élete \\
és egyéb cigerettatípusok}
\author{Dr.~Kiss Géza \\ egyetemi tanár \and
Kovács Alajos \\ nyugdíjas alezredes}
\date{2016. március 9.}


\maketitle

\tableofcontents

%\chapter{Bevezetés}
\section{Irodalmi áttekintés}
Itt jön a szöveg maga\dots

\section{Küldetésünk (Mik a céljaink?)}
Célunk a Schönherz Bázisnál, hogy a velünk kapcsolatba került informatikusok és villamosmérnökök karrierjét folyamatosan támogassuk, követve karrierútjukat és a változó munkaerõ piaci igényeket. Minden életszakaszban, akár pályakezdõ, akár sokat látott mérnökrõl legyen szó, tapasztalatunkra és átfogó piac ismereteinkre támaszkodva olyan tanácsokkal tudunk szolgálni, ami segít a válaszút elõtt álló szakembereknek következõ munkahelyük megtalálásában.
\section{Víziónk (Hogyan képzeljük el mindezt?)}
Hisszük, hogy minden megfelelõ szaktudással rendelkezõ mérnök számára létezik „ideális” munkahely. Mi segítünk ennek megtalálásában, kiválasztásában. A Schönherz környékén lassan 20 éve foglalkozunk azzal, hogy fiatal mérnököket „életre” neveljünk. Valós körülmények között próbálhassák ki magukat, sze\-rezhessenek tapasztalatot és bõvítsék tudásukat gyakorlati ismeretekkel. Ez idõ alatt nagyon sok céggel és fiatal mérnökkel kerültünk kapcsolatba, ismerjük a lehetõségeiket és a problémáikat. Így minden rendelkezésünkre áll, ahhoz hogy a megfelelõ jelöltet mutathassuk be ideális munkahelyeket kínáló partnereinknek.

Nagyon sok munkaerõ-közvetítõ cég keresi a jelöltjeit on-line adatbázisokban és hirdetések útján. Mi a saját kapcsolatrendszerünkre, a Schönherz tevékenységét ismerõkre és az egyetemekrõl, fõiskolákról kikerülõ pályakezdõkre támaszkodunk, ezáltal olyan embereket is elérünk, akiket versenytársaink nem.
\section{Kapcsolat a Schönherz Kollégiummal}
A Schönherz a BME Villamosmérnöki és Informatikai Karának kollégiuma. Itt koncentrálódik a Kar hallgatói közélete és szakmai tevékenysége. A Schönherz nem csak egy egyszerû lakóközösség, jelenlegi és volt lakói rengeteg szakmai és emberi szálon kapcsolódnak egymáshoz. A Kollégium évtizedek óta neveli a szak\-mára, a csapatmunkára, az emberi kapcsolatok fontosságának megértésére az egyetemistákat , aminek köszönhetõen számos volt kollégista ma meghatározó szereplõje az informatikai és villamosmérnöki szakmának. A Schönherz többek között azért képes erre, mert számos hozzá köthetõ szervezet, cég dolgozik azon, hogy a közösségi tevékenységek körülményei adottak legyenek. A Schönherz Bázis is egy ilyen vállalkozás. A Schönherz cégeként egyrészt segítjük a schönherzes közösség tagjait karrierjük elindításában. Másrészt üzleti eredményünk is a hallgatói közéletet és a szakmai fejlõdést támogatja.
\section{„Mérnököt a mérnöktõl”}
A Schönherz Bázist olyan mérnökök alapították, akik tudják, mivel foglalkozik egy mérnök, és milyen szakembereket keresnek a mérnöki vállalkozások. Ha elmondod, milyen munkahelyen, feladatkörben éreznéd jól magad vagy mivel foglalkoztál eddig, akkor azt megértjük, és nem csak a rövidítéseket jegyzeteljük le. Jól kiismerjük magunkat az informatikusok és mérnökök világában, mert ugyanazt a „nyelvet” beszéljük, ez a legnagyobb erõsségünk. Ez a tudás segít minket, az álláskeresõink és a partnercégeink igényeinek lehetõ legjobb megértésében. Az elmúlt években rengeteg szakemberrel kerültünk kapcsolatba, mind munkaadói, mind munkavállalói oldalon. Célunk, hogy a következõ években egyre több informatikussal és villamosmérnökkel találkozzunk, hogy minél többjüknek tudjunk segíteni az elhelyezkedésben vagy az új kolléga megtalálásában. Ezekre a kapcsolatainkra büszkék vagyunk, ápoljuk õket és vigyázunk rájuk. Szóval akár új taggal szeretnétek bõvíteni a csapatotokat, akár épp munkahelyváltásban gondolkozol, ne habozz felvenni velünk a kapcsolatot, mert biztos tudunk segíteni.
\section{Filozófiánk (Milyen alapszabályokhoz igazodunk?)}
\textit{Kiváló álláslehetõségek}

Gondolom ismerõs a kifejezés, legalábbis ezt ígéri mindenki. A bökkenõ, hogy ez mindenkinél mást jelent. Van, aki megbízható, hosszú távú munkahelyet, kellemes munkakörnyezet és versenyképes bért szeretne és van olyan is, aki inkább izgalmas projekteket, tanulási lehetõségeket, változatosságot, utazási lehetõségeket keres. Akad, aki mindezt egyszerre várja a munkaadójától. Mi ezért egyszerre két ügyfélcsoporttal állunk kapcsolatban Egyrészt a munkaadók csoportja, akik új munkatársat keresnek, másrészt a mérnökök, informatikusok csoportja, akik szeretnék építeni a karrierjüket, és szakmailag fejlõdni. Mi olyan partnercégeket keresünk, akikben megvan az a kis plusz, ami miatt érdemes náluk dolgozni.

Ha éppen munkát keresel, gondold át a saját preferenciáidat, így mindenképpen könnyebben tudod rangsorolni a lehetõségeket. Amennyiben tanácstalannak érzed magad, még nem igazodsz el az állást kínáló cégek között, akkor vedd igénybe ingyenes karrier tanácsadásunkat. Mi segítünk megtalálni az utad.

Felsorolások: \dots
\begin{itemize}
\item Állatok
\begin{enumerate}
\item madár ...
\item hal ...
\end{enumerate}
\item növények
\begin{description}
\item egyszikű
\item kétszikű
\end{description}
\end{itemize}

\end{document}