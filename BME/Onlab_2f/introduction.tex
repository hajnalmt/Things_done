%----------------------------------------------------------------------------
\chapter*{Bevezető}\addcontentsline{toc}{chapter}{Bevezetõ}
%----------------------------------------------------------------------------

\hspace{2mm} A Gantt diagramok a projekttervezés alapvető eszközei, segítenek nekünk a feladatok megfelelő elosztásában, monitorozásában. A piacon található alkalmazások közül a Microsoft Project vált a legnépszerűbbé, amit a könnyű kezelhetősége és a benne lévő megannyi funkciónak köszönhet. Ez az alkalmazás vastag kliens technológiára épült így érdekes gondolat lehet ennek az alkalmazásnak a vékony kliens technológiára való átültetése. Jelenleg nagyon kevés olyan vékony kliens alkalmazás van, amely funkcionalitása és robosztussága terén egyáltalán meg tudná közelíteni a Microsoft Projectét. Feladatam a félév során egy ilyen alkalmazás alapjainak a lefektetése volt. Az alkalmazást HTML5-tel, Typescript és Canvas használatának segítségével készítettem.\newline
\indent Az alapvető funkciókat a félév során sikerült megvalósítanom, nyilván funkcionalitást tekintve messze járok még a Microsoft Project-től, de a fejlesztést szeretném a következő félévekben is folytatni. Beszámolómat három pillérre építettem fel. Az első a feladat megvalósításához szükséges irodalomkutatást részletezi, a megvalósított algoritmusokat és a Gantt diagramot mint projektmenedzselési eszközt mutatja be. A második a használt kódolási eszközök bemutatására szolgál, telepítésük módjáról szól, és végül a harmadik fejezetben térek ki az implementációra vázolom az alkalmazás felépítését.