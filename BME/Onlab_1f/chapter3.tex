%----------------------------------------------------------------------------
\chapter{Implementáció}\label{sect:Implement}
%----------------------------------------------------------------------------

Egy ilyen volumenű alkalmazásnak rengeteg követelményt kell teljesítenie, hiszen a cél nem más, mint a Microsoft Project funkcióinak leimplementálása. A következő fejezetben szeretném kifejteni ezeket a követelményeket, továbbá azt, hogy ezeket hogyan teljesítettem.
\section{Felépítés}

\begin{figure}[!ht]
\centering
\includegraphics[width=\textwidth, keepaspectratio]{figures/project1.png}
\caption{Felépítés} 
\label{fig:project1}
\end{figure} 

Szükséges számunkra, egy \textit{hierachia} megvalósítása a feladatok között azaz, hogy lehessen összefogó szülőfeladatokat megadni, alfeladatokat adni egy feladatnak. Fontos, hogy ezek között a feladatok között meg lehessen adni \textit{függőségeket}, azaz meg lehessen mondani egy feladatnak, hogy csak akkor hajthatódjon végre, ha már egy bizonyos másik feladat is végrehajtódott. A függőségekhez tartozik a várakozási idő vagy lag megadásának lehetősége is, azaz, hogy ha két feladat között el kell telnie valamennyi időnek azt meg lehessen adni. Akkor nem csak függőségeket, de \textit{errőforrásokat} is meg lehessen adni egy feladathoz, hiszen tudnunk kell, hogy meg van-e minden erőforrásunk adva az elvégzéséhez. Ezeknek az erőforrásoknak kell legyen egy típusuk és egy számuk, mely a rendelkezésre állásukat reprezentálja. Ezeken kívül szükségünk van egy \textit{munkanaptárra}, hisz ez alapján tudjuk megmondani, hogy milyen munkarend szerint zajlik a munka. Itt fontos, hogy minden naphoz ezt a munkarendet külön be lehessen állítani. Végül pedig már csak egy \textit{ütemezőre} van szükség, ami figyelembe véve a feladatokat megmondja, hogy melyiket mi után kell elvégeznünk optimálisan.


Ezek alapján négy részből állítottam össze a programot, amik mindent tudnak egymásról, és folyamatosan kommunikálnak, lásd \figref{project1}-es ábra:
\begin{itemize}
	\item \textbf{Project:} A feladatokat, függőségeket és összefoglalókat (summaryket) megvalósító osztályok, illetve amik hozzájuk kapcsolódnak.
	\item \textbf{Resources:} Az erőforrások kezelését megvalósító komponens.
	\item \textbf{Scheduler:} Az ütemező osztály, a feladatok ismeretével és a munkanaptár segítségével minden feladathoz hozzárendel egy időtartamot, amikor azt végre kell hajtani.
	\item \textbf{WorkingCalendar:}  Ez az osztály tartja nyilván a munkanapokat egy munkanaptáron keresztül, ez az osztály felelős a munkarendért.
\end{itemize}

%----------------------------------------------------------------------------
\section{Project}
A \figref{project_uml} ábra mutatja a Projekt UML diagramját
\begin{figure}[!ht]
\centering
\includegraphics[width=\textwidth, keepaspectratio]{figures/project_uml.png}
\caption{Project rész UML diagram} 
\label{fig:project_uml}
\end{figure} 
%----------------------------------------------------------------------------
\section{WorkingCalendar}
A \figref{workingcal_uml} ábra mutatja a Working Calendar UML diagramját
\begin{figure}[!ht]
\centering
\includegraphics[width=\textwidth, keepaspectratio]{figures/workingcal_uml.png}
\caption{WorkingCalendar rész UML diagram} 
\label{fig:workingcal_uml}
\end{figure} 

%----------------------------------------------------------------------------
\section{Schedulers, Resources}

A \figref{sched_res_uml} ábra mutatja a Working Calendar UML diagramját
\begin{figure}[!ht]
\centering
\includegraphics[width=\textwidth, keepaspectratio]{figures/sched_res_uml.png}
\caption{A Schedulers és Resources rész UML diagramja} 
\label{fig:sched_res_uml}
\end{figure} 