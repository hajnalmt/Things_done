%----------------------------------------------------------------------------
\chapter*{Bevezető}\addcontentsline{toc}{chapter}{Bevezetõ}
%----------------------------------------------------------------------------

\hspace{2mm} A Gantt diagramok a projekttervezés alapvető eszközei, segítenek nekünk a feladatok megfelelő elosztásában, monitorozásában. Az piacon található alkalmazások közül a Microsoft Project vált a legnépszerűbbé, amit a könnyű kezelhetősége és a benne lévő megannyi funkciónak köszönhet. Ez az alkalmazás vastag kliens technológiára épült és jelenleg nincs párja a piacon. Érdekes gondolat lehet ennek az alkalmazásnak a vékony kliens technológiára való átültetése. Jelenleg nagyon kevés olyan vékony kliens alkalmazás van, amely funkcionalitása és robosztussága terén egyáltalán meg tudná közelíteni a Microsoft Projectét. Feladatam a félév során egy ilyen alkalmazás alapjainak a lefektetése volt. Az alkalmazást HTML5-tel, Typescript és Javascript használatának segítségével készítettem.\newline
\indent Az alapvető funkciókat a félév során sikerült megvalósítanom, nyilván funkcionalitást tekintve messze van még a Microsoft Project-től, de a fejlesztést szeretném a következő félévekben is folytatni. Az első fejezeteben a Gantt alkalmazást szeretném bemutatni, hogy ilyen funkciók és lehetőségek vannak benne. Ehhez a Microsoft Projectet vettem alapul, mint kiinduló alkalmazást. Az azutáni fejezetben ismertetném az alkalmazott technológiákat és eszközöket, végül egy záró fejezetben térek ki a megvalósítás pontos részleteire.