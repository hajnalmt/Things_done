%----------------------------------------------------------------------------
\chapter*{Bevezető}\addcontentsline{toc}{chapter}{Bevezetõ}
%----------------------------------------------------------------------------

\hspace{2mm} Az elmúlt néhány évben hatalmas fejlődés volt a nem-triviális kvantumszámítógépek készítése terén.
Rengeteg start-up dolgozik a technológia üzleti alapokra helyezésével, a NIST standardizálja az új "post-quantum" titkostítási rendszerét, és ipari nagyóriások, mint a Google és a Microsoft, tesznek lépéseket manapság, hogy megvédjék rendszereiket a jövőben lehetséges rosszindulatú kvantumtámadásoktól.
Ezek a széleskörben alkalmazható kvantumszámítógépek már a mi életünk alatt is megszülethetnek.

\indent Az első elektromos számítógépek - a Mark I, Clossus, és ENIAC - drágák, nehezen használhatóak és lassú gépek voltak, elsősorban csak a kormánynak dolgozó kódfeltörőknek és a hadifegyver-készítőknek voltak csak hasznára.
Hasonlóan ezekhez a gépekhez az első kvantumszámítógépek is drágák, lassúak lesznek és alkalmazásuk csak a hagyományos titkosítási rendszerek feltörésére, mint az RSA, valamint a fizaki szimulációk futtatására fog kiterjedni.

\indent Szerencsére számunkra, az idővel a klasszikus számítógépek hardvere olcsóbbá és gyorsabbá vált, és a modern operációs rendszerek által biztosított hardver-absztrakciók, virtuális memória, és időosztás  a számítógéper könnyebben használhatóvá, biztonságosabbá és gyorsabbá tették a mindennap embere számára.

\indent A kérdés tehát, hogy az új kvantumszámítógépeink milyen architektúrát igényelnek és, hogy a milyen operációs rendszer is fog ezeken a gépeken futni?
Ebben a dokumentumban felfedezhetjük a \textit{kvantum operációs rendszer} lehetőségeit, kérdéseit azokkal a részleges válaszokkal, amik egy ilyen új típusú operációs rendszer tervezéséhez kellenek.
\begin{itemize}
\item Milyen új absztrakciók szükségesek egy kvantum operációs rendszer programozásához?
\item A kvantum számítógépek ereje, hogyan tudja megnövelni a klasszikus rendszerek teljesítményét?
\item Hogyan nézne ki egy elosztott kvantum számítógép? És milyen új funkcionalitást tud egy ilyen gép biztosítani?
\end{itemize}

\indent Ez a dokumentum szükségesen (és szégyenszerűen) spekulatív:
Még túlságosan korai egy pontos leírást adni egy csak 20, 50 vagy 100 év múlva létező hardverre...
Egyelőre annyit tehetünk, hogy a meglévő kvantumszámlálási modelleket ráillesztjük egy ilyen széleskörű kvantumszámítógépre, hogy elképzeljük milyen lesz majd az, amikor a jövőben megvalósul.

\indent Az olvasó meglehetősen szkeptikusan nézheti most ezt a dokumentumot, mint egy összefoglalója a negatív eredményeknek: a jelenlegi ismereteink szerint, nincs túl sok olyan dolog most, amit a felhasználó egy kvantumszámítógéppel tud csak megcsinálni és egy hagyományos gyors klasszikus számítógéppel nem.
A célünk hát szimplán, hogy tegyünk egy megjegyzést a kvantum számítógépeknél arra a logikus kérdésre, hogy milyen lenne, ha ott lenne mindannyiunk asztalán egy ilyen mágikus kvantumszámítógép?

\indent A dokumentumban három lehetséges architektúrát vázolunk fel, haladva a legkevésbé merésztől a legmerészebb elképzelésig:
Először is a kvantum FPGA-król ejtünk szót, utána a kvantum x86-os számítógépekről, és végül a kvantum elosztott rendszerekről.
Mindegyiknél fogjuk tárgyalni, hogy a gép hogyan kezelné a legalapvetőbb feladatokat, mint a fuzz tesztelés, CPU ütemezés és párhuzamos programozás.
Beszélünk arról is, hogy az egyes architektúrák hogyan felelnek meg a rendszerszintű kihívásoknal.
