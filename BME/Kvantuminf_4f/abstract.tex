%----------------------------------------------------------------------------
% Abstract in hungarian
%----------------------------------------------------------------------------
\chapter*{Összefoglaló}\addcontentsline{toc}{chapter}{Összefoglaló}

\hspace{2mm} Ez a dokumentum a Hálózati Rendszerek és Szolgáltatások tanszék Kvantuminformatika és kommunikáció nevezetű tantárgy házi feladatának teljesítésére készült.
A házi feladat egy a kvantumkommunikációs közösségi könyv számára írt fejezet megalkotása volt.
A lehetséges témák közül a QOS-t (Quantum Operating Systems) azaz a Kvantum Operációs Rendszer bemutatását választottam.

\indent Hogyha a kvantum-komputerek mindennapossá válnak világunkban, akkor az operációs rendszernek is tudnia kell biztosítani az újszerű absztrakciót ennek a bizar új harwer erejének megszelidítéséhez.
Ezen dokumentumban végigvesszük a problémákat, ami ebből a meglehetősen nagy feladatból fakad.
Ezen kívül demonstráljuk, hogy, hogy ezek a gépek milyen meglepő sebességnövekedést okoznak sok mindennapos rendszerfeladat ellátására, mint a unit-testing vagy a CPU ütemezés.
Magának az alkotásnak az alapját Henry Corrigan Gibbs, David J. Wu, és Dan Boneh Stanfordi kutatók 2017-ben megjelent Quntum Operating Systems című cikkük jelenti.