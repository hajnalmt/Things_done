%----------------------------------------------------------------------------
\chapter{Elosztott rendszerek}\label{sect:distributed}
%----------------------------------------------------------------------------
\hspace{2mm} A legutolsó architektúra amit megnézünk az a legspekulatívabb is egyben.
Tegyük fel, hogy van egy hálózatunk kvantumszámítógépekből, melynek összeköttetései, linkjei képesek qubitek átvitelére.
Ebben a fejezetben felfedezzük, mint is tudnánk elérni egy ilyen rendszerrel.
%----------------------------------------------------------------------------
\section{A szupersűrű kódolás előrevetítése}
\hspace{2mm} Rengeteg hálózati terhelés burstös, kis ideig nagymennyiségű.
Például egy sima böngésző az ideje nagy részében nem hajt végre semmilyen kommunikációt, de minden oldal betöltéskor megabájt nagyságrendű adatot probál letölteni olyan gyorsan, ahogyan lehetséges.

\indent Egy technika, annak érdekében hogy a hálózat kevésbé legyen ilyen burstös, hogy összeköttetés prefetch-elést alkalmazunk, azaz megpróbálja browser megmondani, hogy mi lesz a következő tartalom amit a felhasználó le akar majd tölteni és előre leszedi a háttérben azt, még mielőtt a user rákattintana.
Egy kvantumjelenség az ún. \textit{szupersűrű kódolás} lehetővé teszi a kvantum kliensnek, hogy letöltse, prefetchelje az adatot a kvantumszerverére, úgy hogy a kliensnek még ötlete sincs róla, hogy milyen adatra lesz szüksége a jövőben.
\vspace{2mm}
\textbf{Tény 3}  \textit{(szupersűrű kódolás) Ha egy kliens és egy szerver egy adott összefonódott qubiten osztozik (előre megosztott qubit), akkor a szerver elküldhet két klasszikus bitből álló információt a kliensnek egy szimpla qubit elküldésével.}
\vspace{2mm}
\indent Ahhoz, hogy implementáljuk a prefetchelést a szupersűrű kódolást használva a szervernek folyamatosan kell összefonódott qubiteket gyártania, melyekből egyet elküld a kliensnek.
Amikor a klens le szeretne tölteni valamilyen adat a gépére, akkor használhatja az összefonódott qubitjét, amin a szerverrel osztoznak, hogy átküldjön adatot \textit{kétszer} olyan nagy bitszámmal, mint amekkora a hálózati összeköttetés kapacitása.
A szupersűrű kódolás megnövelheti a hálózat teljesítményét még akkor is amikor konstans sebességgel tud csak adatot letölteni a szerverről.
Mondjuk azt, hogy a kliens és a szerver egy olyan hálüzati kábellel van összekötve, amelyen az átvitel sebesség 100 Mbps mindkét irányban.
Ezt a szupersűrű kódolást használva, a klens a szerverről \textit{200Mbps}-el tudna adatot letölteni.
Ahhoz, hogy ez sikerüljön összefonódott qubiteket kell küldenie a kliensnek a szerver felé 100Mbps-es sebességel, a szerver pedig visszaküldeni a kliensnek kódolt qubiteket 100Mbps-es sebességgel.
A kliens így a csatornából 200Mbps-es sebességgel tudna információt szerezni, ezzel egy 100Mbps-es kétirányú kommunikációt 200Mbps-es iránymentes kommunikációvá változtatva.
 

%----------------------------------------------------------------------------
\section{Távoli keresés}
\hspace{2mm} Számítógépek kvantum linkkekkel összekötve futtathatnak Grover algoritmust a hálózaton.
Csak, hogy lássunk egy lehetséges alkalmazását ennek az ötletnek: Mondjuk azt, hogy egy kliens szeretne keresni egy speciális elemet egy nagy adattartományon ami $N$ elemből áll $(x_i,...,x_N)$ egy adott szerveren tárolva (pl. lehet szó akár terabájtnyi videó adatról.)
Ebben az esetben lehetne a kliensnek egy osztályozója: $f:\{0,1\}\rightarrow \{0,1\}$ és szeretne egy indexet találni a szervern $i^*$-ot amelyre igaz, hogy $f(x_{i^*})=1$.
A kliens nem szeretné letölteni az egész adatállományt a szerverről és nem is szeretné feltölteni a szerverére az $f$ osztályozóját, lehet hogy ez az osztályozó túl nagy, hogy elküldje vagy van benne valamilyen titkos input, algoritmus.

\indent Egy ilyen esetben tudna akkor a szerveren futtatni egy Grover algoritmust a kliens a hálózaton keresztül, hogy találjon egyező értéket $x_{i^*}$, miközben a szerverrel csak alig $\Theta(\sqrt{N})$ qubitet kellett cserélnie az eredeti klasszikus $\Theta(N)$ bit helyett.
Persze a számolást nem tudjuk megspórolni a kliens és a szerver oldalon, a kliensnek szükséges az osztályozóját kb. $\sqrt{N}$-el lefuttatnia és szervernek is szükséges, hogy lefuttasson kb. $\sqrt{N}$-szer lefuttatnia egy qRAM lekérdezést.
Mindenesetre a hálózati szállításból nyert költség lehet, hogy arányaiban megérték ezt a számolási többletet.