%----------------------------------------------------------------------------
\chapter{Kvantum FPGA-k}\label{sect:FPGA}
%----------------------------------------------------------------------------
\hspace{2mm} Egy elfogadható felépítése az általánoscélú kvantumszámítógépek első generációjának, hogy egy klasszikus processzor van összekötve klasszikus buszon keresztül egy kvantumalapú FPGA-hoz ún. qFPGA-hoz (quantum field programmable gate array).

\indent Mint egy hagyományos FPGA, a qFPGA egy periférián keresztül tud csatlakozni egy klasszikus CPU-hoz (Lásd: \figref{fpga}es ábra).
Mielőtt használni lehetne a qFPGA-t a programozónak előszöt ki kell találnia egy kvantumáramkört $C$-t, mely implementálja az általa megoldani kívánt problémát.
A kvantumáramkör egy klasszikus bit-alapú áramkörhöz hasonlí, azzal a kivétellel, hogy alapul kvantumkapukat használ. (pl: Toffoli és Hadamard kapukat) nem pedig hagyományosakat (ÉS, VAGY, NEM).

\begin{figure}[!ht]
\centering
\includegraphics[width=0.5\textwidth, keepaspectratio]{figures/fpga.png}
\caption{Egy klasszikus számítógép egy kvantum FPGA-hoz kötve. A klasszikus processzor beprogramozza a qFPGA-t a kvantumáramkör leírásával $C:\{0,1\}^n \rightarrow {0,1}^n$. A CPU ezután elküldi a bemeneteket $x_i \in \{0,1\}^n$ és visszakapja a kvantumáramkör kimeneteket $C(x_i)$ minden $x_i$-re.}
\label{fig:fpga}
\end{figure}

\indent A programozó ezután elküldi a kvantumáramkör leírását egy klasszikus buszon keresztül. Az áramkör felprogramozása után a CPU már küldheti is a bemeneteket a qFPGA számára.
Egy klasszikus feldolgozó egység betáplálja a qFPGA minden kvantum kapujának eredményét qFPGA állapotregiszterébe szekvenciálisan.
Végül a qFPGA visszaküldi a CPU-nal az áramkör kimeneteket, $C(x_i)$-ket minden $x_i$-re alkalmazva.
A bemenetek, $x_i$ és kimenetek, $C(x_i)$ mind klasszikus bitsorozatok, így a CPU és a qFPGA egy klasszikus buszon keresztül is kommunikálhatnak.

\indent A jelenlegi tudásunk szerint még nagyon-nagyon messze vagyunk, bármi hasonló, mint egy ilyen kvantum FPGA készítésétől. A jelenlegi generációja a kvantumszámítógépeknek csak pár qubiten tud egyelőre csak múveleteket végezni, így nagyságrendekkel le vagyunk egy ilyen FPGA-hoz hasonló eszköztől, ami esetleg nagy skálán tud kvantumszámításokat végezni.
Attól viszont még el tudunk képzelni egy ilyen eszközt. Mire lennénk vele képesek?

\indent Egy nyilvánvaló alkalmazása lenne, hogy Schor algoritmusát futtathatnánk rajta egész számok faktorizációjához és diszkrét logaritmusok számításához.
Bár mire oda jutunk, hogy qFPGA-k elérhetőek lesznek, a világ már rég átvált a kvantumtámadásokra felkészített titkosításokra.
Tehát ehelyett azt várjuk, hogy az elsődleges mindennapi használata ezeknek az eszközöknek a Grover keresések futattása lesz olyan programokon, melyek így nagyon kicsiny áramkörökre is lefordíthatóvá válnak.
Két példát adunk az alkalmazásokra, az egyik az ún. code fuzzing  a másik pedig a jelszó feltörés.
%----------------------------------------------------------------------------
\section{Code fuzzing}

\indent A code fuzzing ötlete mögött az van, hogy egy darabka kódrészletet futtatunk le nagy számú határesetszerű bemenettel, hogy megnézzük milyen esetekben nem működik a program rendeltetésszerűen.
Erre az alkalmazásra lehetne qFPGA-t használni így biztosítva, hogy az egyes fordítók trükkös optimalizációi nem rontanak a program  működésének helyességén.
Például, egy programozónak  lehet két bit-alapú áramköre $n$-bites egész számok összeszorzására: egy alap $M_{slow}$ natív C kódból fordítva, és egy optimalizált verzió $M_{fast}$.
A programozó ekkor készíthet egy áramkört $C:\{1,...,N\}^2 \rightarrow \{0,1\}$, ami kiszámolja

$$C(x,y)=
		\begin{cases}
		1 & \text{if } M_{slow}(x,y)\neq M_{fast}(x,y)\\
		0 & \text{otherwise}
		\end{cases}
$$

A qFPGA-n így futhat Gorver keresése $C$-n, a kiemeneti különbségekre.
\indent A qFPGA-t használva a programozó megbizonyosodhat róla, hogy a két szorzás rutin ugyanúgy viselkedik a $2^{64}$-féle lehetséges bemeneten úgy, hogy mindösszesen $2^{32}$-szer kellett meghívni $M_{slow}$-t és $M_{fast}$-ot.
Továbbá, hogyha egy hiba előjön az optimalizált áramkörön $k$-szor az $N$ bemenetből, akkor a qFPGA segítségével a programozó körülbelül $\sqrt{N/k}$ idő alatt meghatározhatja annak a bemenetnek az indexét, amely a hibát okozta.
%----------------------------------------------------------------------------
\section{Password cracking}
\indent Abban az időben, amiben a kvantum FPGA-k elérhetőek lesznek, a "kvantumapokalipszis" már rég megölte a legtöbb mai kvantumsebezhető kriptográfiai algoritmusokat.
Azonban, mint a makacs rovarok, akik túlélték a dinoszauruszok kipusztulását, azt várjuk, hogy az ember által választott mindenütt jelenlévő jelszavakra (bár talán kicsit vonakodva), de az igény megmarad a kvantum korban is.

\indent Egy néhány becstelen alkalmazása a qFPGA-nak, amikről nem beszéltünk, hogy fel tudnak majd törni könnyen jelszavakat.
Egy ilyen feltörés során a támadónak van egy kriptografikus hash-je $h$ a felhasználónak a jelszaváról, továbbá meg van neki a $H$ hash funkció is amivel titkosították és megszeretné találni azt a $p$ jelszót amire igaz, hogy $h=H(p)$.
Mivel a felhasználók gyakran csak a jelszavak egy népszerű készletéből választanak $D$, a támadónak kitalálhatja a jelszót úgy, hogy megnézi ezeknek a népszerű jelszavaknak a hash-ét és megnézi, hogy van e olyan szó $d* \in D$, hogy $h=H(d*)$.

\indent Amíg egy hagyományos támadónál a feltörés ideje nagyjából lineárisan megegyezik a szótár méretével $D$-vel, addig egy qFPGA háttérel rendelkezőnek csak $\sqrt{|D|}$-nyiszer kell meghívnia a hash funkciót.
Csak, hogy konkrét számokról is beszéljünk: összesen 95 darab nyomtatható ASCII karakter van, azaz egy 10 betűs jelszónál $95^{10}\approx 2^{66}$ lehetséges variáció létezik. Hogyha minden egyes felhasználó egy irreálisan magas entriópiával rendelkező szórásból is választott, egy támadó visszafejtheti a hash funkció $H$ mintegy $\sqrt{2^{66}}=2^{33}$ meghívásával.
Egy alacsony költségű qFPGA, tehát könnyen és gyorsan megoldaná nekünk ezt a feladatot.

\indent A mai modern "memória-taró" jelszó hashelési funciók , nagyon nagy klasszikus áramköröket igényelnek, így talán kevésbé lesznek kitéve ezeknek a támadásoknak.
Az viszont egy érdekes nyitott kérdés, hogy léteznek-e esetleg a Grover keresésnél hatékonyabb speciális kvantum támadások, melyekkel talán fel lehetne törni ezeket a funkciókat is.