% Készítette: Hajnal Máté
% Az Elte Programtervező Informatikus szakához tartozó Programozás tantárgy feladatainak kidolgozása

\documentclass[12pt,twoside,a4paper]{article}	%Article dokumentum
\usepackage[utf8]{inputenc}						%UTF-8-as kódolás
\usepackage{t1enc}								%Furcsa betűk
\usepackage{fancyhdr}
%texlive-lang-hungarian package should be installed!
\usepackage[english,magyar]{babel}				%Magyar nyelv 
\usepackage{mathptmx}							%Times New Roman font

\fancyhead{}
\fancyfoot{}
\fancyhead[L]{Programozás - Visszavezetéses feladatok}
\fancyhead[R]{Készítette: Hajnal Máté}
\fancyfoot[L]{Hibák előfordulhatnak a kidolgozásban}
\pagestyle{fancy}

\begin{document}
\pagenumbering{arabic}

\textbf{Egyetlen programozási tétellel megoldható feladatok}
\begin{enumerate}
\item{Vezesse vissza összegzésre az alábbi feladatokat:}
	\begin{enumerate}
	\item{Két nem-negatív szám szorzatának kiszámolása összeadásokkal.}
	\item{Faktoriális kiszámítása.}
	\item{Szám hatványának kiszámolása összeszorzásokkal.}
	\item{Mennyi egy egydimenziós tömbbeli (vektorbeli) valós számok abszolút értékeinek összege?}
	\item{Mennyi két azonos hosszú, valós számokból álló vektor skaláris szorzata?}
	\item{Számoljuk meg egy egész számokat tartalmazó vektorban a páros elemeket!}
	\item{Van-e egy n darab egész számokat tartalmazó vektorban páros szám?}
	\item{Egy egész számokat tartalmazó vektor minden eleme páros-e?}
	\item{Válogassuk ki egy egész számokat tartalmazó vektorból a páros elemeket egy sorozatba!}
	\item{Válogassuk ki egy egész számokat tartalmazó vektorból a páros elemeket egy halmazba!}
	\item{Természetes számokat tartalmazó tömb maximális elemének meghatározása.}
	\end{enumerate}
\end{enumerate}
\end{document}