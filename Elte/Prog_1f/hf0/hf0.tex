% Készítette: Hajnal Máté
% Az Elte Programtervező Informatikus szakához tartozó Programozás tantárgyhoz tartozó házi feladatom.
% A feladatok közül a 11-es

\documentclass[12pt,twoside,a4paper]{article}	%Article dokumentum
\usepackage[utf8]{inputenc}						%UTF-8-as kódolás
\usepackage{t1enc}								%Furcsa betűk
\usepackage{fancyhdr}							%For footer and header
%texlive-lang-hungarian package should be installed!
\usepackage[english,magyar]{babel}				%Magyar nyelv 
\usepackage{mathptmx}							%Times New Roman font
\usepackage{enumitem}							%For enumerations
\usepackage{url}								%For URL-s
\usepackage[headheight=56pt]{geometry}			
\usepackage[fleqn]{amsmath}						%Mathematical expressions
\usepackage{amsfonts}							%Mathematical fonts
%Packages for Table handling
\usepackage{array}
\usepackage{tabu}
\usepackage{multirow}
\usepackage{tabularx}
\usepackage{tikz}

\sloppy
\fancyhead{}
\fancyhead[C]{\textbf{0.beadandó feladat/11.feladat}}
\fancyhead[R]{\today}
\fancyhead[L]{Hajnal~Máté \newline RJBSCJ \newline \url{hajnalmt@inf.elte.hu} \newline 5.csoport}
\pagestyle{fancy}

\newcommand{\fejezet}[1]{\noindent \textbf{\textit{\large #1 \vspace{5mm}}}}

\begin{document}
\pagenumbering{arabic}
%%%%%%%%%%%%%%%%%%%%%%%%%%%%%%
\fejezet{Feladat}\\
\textit{A föld felszínének egy vonala mentén egyenlő távolságonként megmértük a terep tengerszint feletti magasságát és a mért értékeket egy tömbben tároljuk. Keressük meg a legmagasabb völgyet a mérési sorozatban!}
\vspace{5mm} \\
%%%%%%%%%%%%%%%%%%%%%%%%%%%%%%
\fejezet{Specifikáció}\\
Az értékeket egy tömbben tároljuk.
\begin{flalign*}
&A=(t:\mathbb{R}^n,~max:\mathbb{R},~ind:\mathbb{N},~l:\mathbb{L})\\
&Ef=(t=t')\\
&Uf=(Ef~\wedge~\max_{i=2,~t[i]<t[i-1],~t[i]<t[i+1]}^{i=n-1}=t[i])
\end{flalign*}
%%%%%%%%%%%%%%%%%%%%%%%%%%%%%%
\fejezet{Algoritmus}\\
A feladatot a feltételes maximumkeresés tételére vezetjük vissza.
%%%%% Megfeleltetés %%%%%
\begin{flalign*}
&m..n~\sim~2..n-1\\
&\beta~\sim~t[i]<t[i-1],~t[i]<t[i+1]\\
&f(i)~\sim~t[i]
\end{flalign*}
%%%%% Table %%%%%
\begin{tabular}{|m{1em}|m{7em}|m{7em}|m{3em}|m{8em}|}
\hline
	\multicolumn{5}{|c|}{$l:=hamis$}\\
\hline
	\multicolumn{5}{|c|}{$i=2..n-1$}\\
\cline{2-5}
	& $t[i]>t[i-1]~\vee \newline t[i]>t[i+1]$ &
	\multicolumn{2}{c|}{$l~\wedge~t[i]<t[i-1]~\wedge~t[i]<t[i+1]$} &
	$\neg l~\wedge~t[i]<t[i-1] \newline \wedge~t[i]<t[i+1]$ \\
\cline{2-5}
	 \multirow{2}{*}{} & \multirow{2}{*}{SKIP} & \multicolumn{2}{c|}{$max < t[i]$} & $l,~max,~ind :=$  \\\cline{3-4} & & $max,~ind:=t[i],~i$ & SKIP & $igaz,~t[i],~i$ \\
\hline
\end{tabular}\\
\fejezet{Implementáció}\\
\large{Adattípusok megvalósítása}\\
A kódoláskor a t tömböt \texttt{vector<int>}-ként deklaráltam, amelynek mérete \texttt{t.size()} alakban érhető el. Mivel a vektor C++-ban 0-tól indexelődik, így a tervbeli 2-től n-1-ig tartó ciklus az esetünkben \texttt{1-től t.size()-2-ig} fog tartani. A struktogramm kódja így:
\texttt{
	\begin{tabbing}
		\hspace{2cm}\= l=false \+\\
		for (int i=1; i < t.size()-1; ++i) \{\\
		\hspace{1cm}\= if (t[i-1] > t[i] \&\& t[i+1] > t[i] \&\& l) \{ \+\\
			\hspace{1cm}\=if (t[i] > max) \{ \+\\
			\hspace{1cm}\=max = t[i]; \+\\
			ind = i-1;\-\\
			\}\-\\
		\}\\
		else if (t[i-1] > t[i] \&\& t[i+1] > t[i]) \{ \+\\
			l = true;\\
			max = t[i];\\
			ind = i-1;\-\\
		\} \-\\
	\}
	\end{tabbing}
}
\large{Bemenő adatok formája}\\
\fejezet{Tesztelési terv}
\end{document}
