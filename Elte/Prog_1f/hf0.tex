% Készítette: Hajnal Máté
% Az Elte Programtervező Informatikus szakához tartozó Programozás tantárgyhoz tartozó házi feladatom.
% A feladatok közül a 11-es

\documentclass[12pt,twoside,a4paper]{article}	%Article dokumentum
\usepackage[utf8]{inputenc}						%UTF-8-as kódolás
\usepackage{t1enc}								%Furcsa betűk
\usepackage{fancyhdr}							%For footer and header
%texlive-lang-hungarian package should be installed!
\usepackage[english,magyar]{babel}				%Magyar nyelv 
\usepackage{mathptmx}							%Times New Roman font
\usepackage{enumitem}							%For enumerations
\usepackage{url}								%For URL-s
\usepackage[headheight=56pt]{geometry}			
\usepackage{amsmath}							%Mathematical expressions
\usepackage{amsfonts}							%Mathematical fonts

\sloppy
\fancyhead{}
\fancyhead[C]{\textbf{0.beadandó feladat/11.feladat}}
\fancyhead[R]{\today}
\fancyhead[L]{Hajnal~Máté \newline RJBSCJ \newline \url{hajnalmt@inf.elte.hu} \newline 5.csoport}
\pagestyle{fancy}

\newcommand{\fejezet}[1]{\noindent \textbf{\textit{\large #1 \vspace{5mm} \newline}}}

\begin{document}
\pagenumbering{arabic}
\fejezet{Feladat}
\textit{A föld felszínének egy vonala mentén egyenlő távolságonként megmértük a terep tengerszint feletti magasságát és a mért értékeket egy tömbben tároljuk. Keressük meg a legmagasabb völgyet a mérési sorozatban!}
\vspace{5mm} \\
\fejezet{Specifikáció}
Az értékeket egy tömbben tároljuk.
\begin{align*}
&A=(t:\mathbb{R}^n,~max:\mathbb{R},~ind:\mathbb{N},~l:\mathbb{L})\\
&Ef=(t=t')\\
&Uf=(Ef~\wedge~\max_{i=2,~t[i]<t[i-1],~t[i]<t[i+1]}^{i=n-1}=t[i])\\
\end{align*}
\fejezet{Algoritmus}

\end{document}