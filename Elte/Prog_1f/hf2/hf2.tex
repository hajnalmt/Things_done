% Készítette: Hajnal Máté
% Az Elte Programtervező Informatikus szakához tartozó Programozás tantárgyhoz tartozó házi feladatom.
% A feladatok közül a 11-es

\documentclass[12pt,a4paper]{article}			%Article dokumentum
\usepackage[utf8]{inputenc}						%UTF-8-as kódolás
\usepackage{t1enc}								%Furcsa betűk
\usepackage{fancyhdr}							%For footer and header

%LANGUAGE-FONT
%texlive-lang-hungarian package should be installed!
\usepackage[english,magyar]{babel}				%Magyar nyelv 
\usepackage{mathptmx}							%Times New Roman font
\usepackage[shortlabels]{enumitem}							%For enumerations
\usepackage{url}								%For URL-s
\usepackage[headheight=56pt]{geometry}			%For the heading gap

%MATHEMATICAL EXPRESSIONS
\usepackage[fleqn]{amsmath}						%Mathematical expressions
\usepackage{amsfonts}							%Mathematical fonts
\usepackage{bm}									%Bolding

%TABLES, TABULATING, DRAWING
\usepackage{array}
\usepackage{tabu}
\usepackage{multirow}
\usepackage{tabularx}
\usepackage{tikz}
\usepackage{stuki}

%HEADER
\sloppy
\fancyhead{}
\fancyhead[C]{\textbf{1.beadandó feladat/11.feladat}}
\fancyhead[R]{\today}
\fancyhead[L]{Hajnal~Máté \newline RJBSCJ \newline \url{hajnalmt@inf.elte.hu} \newline 5.csoport}
\pagestyle{fancy}

%Fejezet stílus deklaráció
\newcommand{\fejezet}[1]{\noindent \textbf{\textit{\large #1 \vspace{5mm}}}}

%DOCUMENT
\begin{document}
	\pagenumbering{arabic}
	
	%% Feladat fejezet %%
	\fejezet{Feladat}\\
	\textit{Egy étteremben a pincérek által felvett rendeléseket egy szekvenciális input fájlban tartják nyilván az ételek neve, azon belül a rendelések időpontja szerint rendezett formában. Feltehetjük, hogy a fájl nem üres. A tárolt adatok: a rendelt étel neve, a rendelés időpontja, rendelt adagok száma, egy adag ára. Melyik étel hozta az étteremnek a legtöbb bevételt (összesített darab*egységár)?}
	\vspace{5mm}

	%% Specifikáció fejezet %%
	\fejezet{Specifikáció}\\
	A feladat állapottere többféleképpen felírható
	% A feladat állapottere
	\begin{flalign*}
		A=(f:Infile(&\mathbb{K}),~cout:Outfile(\mathbb{K}))\\
		A=(f:Infile(&Sor),~cout:Outfile(Sor))\\
		A=(f:Infile(&Rendeles), cout:String)\\
		&Rendeles=\textbf{rec}(nev:String, ido:\mathbb{N}, adag:\mathbb{N}, ar:\mathbb{N})\\
	\end{flalign*}
	Számunkra a legideálisabb már egy olyan felsorolón dolgozni, ami rendelés nevű rekordokat dolgozni, melyek a rendelések nevét és hozzájuk tartozó bevételeket tartalmazzák. Erre a felsorolóra kell nekünk egy maximum keresést írni.\\
	% Összegzés, összefűzés
	A maximumkeresés:
	\begin{flalign*}
	&A=(t:Enor(Rendeles),~cout:String,~max:\mathbb{N},~elem:Rendeles)\\
	&\hspace{25mm}Rendeles=\textbf{rec}(nev:String, bevetel:\mathbb{N})\\
	&Ef=(t=t'~\wedge~|t'|>0~\wedge~t.azon\uparrow)\\
	&Uf=((max,~elem)=\max_{e\in t'}{(e.bevetel)}~\wedge~cout=elem.nev)
	\end{flalign*}
	%% Absztrakt prgram fejezet %%
	\fejezet{Absztrakt~program}
	\begin{center}
	\begin{tabular}{|lll|}
		\hline
		\multicolumn{3}{|c|}{\textbf{Maximum~keresés}}\\
		\hline
		t & $\sim$ & \textit{Rendelések bevételeivel való felsorolója}\\
		+,0 & $\sim$~ & \textit{+,0}\\
		f(e) & $\sim$ & \textit{e.bevetel}\\
		\hline
	\end{tabular}
	\end{center}

	\noindent\hfill
		\begin{stuki}[12cm]
			\stm{t.First()}
			\stm{max,~elem:=t.Current().bevetel,~t.Current()}
			\stm{t.Next()}
			\begin{WHILE}{4}{\stm{\lnot t.End()}}
				\begin{IF}[70]{2}{\stm{t.Current().bevetel>max}}
					\stm{max,~elem:=t.Current().bevetel,~t.Current()}
				\ELSE
					\stm{SKIP}
				\end{IF}
				\stm{t.Next()}
			\end{WHILE}
			\stm{cout:=elem.nev}
		\end{stuki}
		\vspace{5mm}
	\textbf{Rendelések bevételeivel való felsorolója:}
	\vspace{5mm}\\
		\begin{tabular}{|l|l|}
			\hline
			\textit{enor(Rendeles)}& \textit{First(),Next(),Current(),End()}\\
			\hline
			\textit{f:Infile(nev:String,ido:$\mathbb{N}$,adag:$\mathbb{N}$,ar:$\mathbb{N}$) }& \textit{First()$\sim$ st, $akt_f$, f:read;Next()}\\
			\textit{st:Status} & \textit{Next()$\sim$ lsd.külön}\\
			\textit{elso:String} & \textit{Current()$\sim$ akt}\\
			\textit{akt:Rendeles} & \textit{End() $\sim$ st=abnorm}\\
			\textit{$akt_f$:rec(nev:String,ido:$\mathbb{N}$,adag:$\mathbb{N}$,ar:$\mathbb{N}$)} & \\
			\hline
		\end{tabular}
	\vspace{5mm}\\
	\indent A First művelettel beolvastunk a szekvenciális input fájlunkból egy rendelés elemet, melyet az $akt_f$ rekordban tárolunk. A beolvasás sikerességét a státusz jelzés jelzi nekünk. \\
	\indent A Next művelet több feladatot, kell ellásson. Addig kell olvasson a szekvenciális input fájlból, amíg a rendelés nevek megegyeznek, vagy véget nem ér a fájl. Ehhez először egy elso nevű változóban eltároljuk az aktuális rendelés nevét, továbbá ehhez hasonlóan egy akt nevű változóban eltároljuk a szükséges rendelés paramétereket, (név és bevétel=adag*ár). Az összegzést, úgy valósítjuk meg, hogy az $akt_f$-be beolvasott input fájl értékek segítségével (az adag*ár értékekkel) az aktuális felsoroló elem $akt$ bevétel nevű változóját növelgetjük. A ciklus akkor marad abba, ha a beolvasás nem sikerült, vagy új rendelésünk van.

	\begin{flalign*}	
	&A^{Next}=(f:Infile(nev:String,ido:\mathbb{N},adag:\mathbb{N},ar:\mathbb{N}),elso:String,st:Status,\\
	&\hspace{30mm}act_f:rec(nev:String,ido:\mathbb{N},adag:\mathbb{N},ar:\mathbb{N}),akt:Rendeles)\\
	&Ef^{Next}=(f=f^1~\wedge~st=st^1~\wedge~akt_f=akt_f^1)\\
	&Uf^{Next}=((st=norm\rightarrow elso=akt_f^1.nev\\
	&\hspace{30mm}\wedge~akt.bevetel^2=akt_f^1.adag*akt_f^1.ar\\
	&\hspace{30mm}\wedge~akt.nev^2=akt_f^1.nev)~\wedge~\\
	&(st^2,{akt_f}^2,f^2)=((akt.nev,akt.bevetel)=\sum\limits_{akt_f\in {akt_f}^1,f^1}^{elso=akt_f.nev \wedge st=norm}{(akt_f.nev,} \\
	&\hspace{10mm}akt.bevetel+akt_f.adag*akt_f.ar)\\
	\end{flalign*}	

	\noindent\hfill
		\begin{stuki}[12cm]
			\begin{IF}[70]{2}{\stm{st=norm}}
				\stm{elso,akt.nev,akt.bevetel=\\
					akt_f.nev,akt_f.nev,akt_f.adag*akt_f.ar}
			\ELSE
				\stm{SKIP}
			\end{IF}
			\begin{WHILE}{2}{\stm{elso=akt_f.nev \wedge st=norm}}			
				\stm{st,akt_f,f:read}
				\stm{akt.nev,akt.bevetel:=akt_f.nev,akt.bevetel+akt_f.adag*akt_f.ar}
			\end{WHILE}
		\end{stuki}
		\vspace{5mm}
	%% Absztrakt prgram fejezet %%
	\fejezet{Implementáció}		
\end{document}