% Készítette: Hajnal Máté
% Az Elte Programtervező Informatikus szakához tartozó Programozás tantárgyhoz tartozó házi feladatom.
% A feladatok közül a 11-es

\documentclass[12pt,a4paper]{article}			%Article dokumentum
\usepackage[utf8]{inputenc}						%UTF-8-as kódolás
\usepackage{t1enc}								%Furcsa betűk
\usepackage{fancyhdr}							%For footer and header

%LANGUAGE-FONT
%texlive-lang-hungarian package should be installed!
\usepackage[english,magyar]{babel}				%Magyar nyelv 
\usepackage{mathptmx}							%Times New Roman font
\usepackage[shortlabels]{enumitem}							%For enumerations
\usepackage{url}								%For URL-s
\usepackage[headheight=56pt]{geometry}			%For the heading gap

%MATHEMATICAL EXPRESSIONS
\usepackage[fleqn]{amsmath}						%Mathematical expressions
\usepackage{amsfonts}							%Mathematical fonts
\usepackage{bm}									%Bolding

%TABLES, TABULATING, DRAWING
\usepackage{array}
\usepackage{tabu}
\usepackage{multirow}
\usepackage{tabularx}
\usepackage{tikz}
\usepackage{stuki}

%HEADER
\sloppy
\fancyhead{}
\fancyhead[C]{\textbf{1.beadandó feladat/11.feladat}}
\fancyhead[R]{\today}
\fancyhead[L]{Hajnal~Máté \newline RJBSCJ \newline \url{hajnalmt@inf.elte.hu} \newline 5.csoport}
\pagestyle{fancy}

%Fejezet stílus deklaráció
\newcommand{\fejezet}[1]{\noindent \textbf{\textit{\large #1 \vspace{5mm}}}}

%DOCUMENT
\begin{document}
	\pagenumbering{arabic}
	
	%% Feladat fejezet %%
	\fejezet{Feladat}\\
	\textit{Egy étteremben a pincérek által felvett rendeléseket egy szekvenciális input fájlban tartják nyilván az ételek neve, azon belül a rendelések időpontja szerint rendezett formában. Feltehetjük, hogy a fájl nem üres. A tárolt adatok: a rendelt étel neve, a rendelés időpontja, rendelt adagok száma, egy adag ára. Melyik étel hozta az étteremnek a legtöbb bevételt (összesített darab*egységár)?}
	\vspace{5mm}

	%% Specifikáció fejezet %%
	\fejezet{Specifikáció}\\
	A feladat állapottere többféleképpen felírható
	% A feladat állapottere
	\begin{flalign*}
		A=(f:Infile(&\mathbb{K}),~cout:Outfile(\mathbb{K}))\\
		A=(f:Infile(&Sor),~cout:Outfile(Sor))\\
		A=(f:Infile(&Rendeles), cout:String)\\
		&Rendeles=\textbf{rec}(nev:String, ido:\mathbb{N}, adag:\mathbb{N}, ar:\mathbb{N})\\
	\end{flalign*}
	Számunkra a legideálisabb rendelés nevű rekordokkal dolgozni egy felsorolóban. A feladatot két részfeladatra bontással oldjuk meg. Az első részben egy szekvenciális input fájlban összegezzük minden ételre a bevételt ezt ki is írjuk egy szekvenciális inputfájlba, a másodikban megkeressük a maximumbevételt hozót. Tehát egyfelsorolókra való összegzésről (azon belül is egy sima összegzésről) és egy maximum keresésről van szó.\\
	% Összegzés, összefűzés
	Az összegzés:
	\begin{flalign*}
	&A=(x:Enor(Rendeles_1),~y:Enor(Rendeles_2))\\
	&\hspace{25mm}Rendeles_1=\textbf{rec}(nev:String, ido:\mathbb{N}, adag:\mathbb{N}, ar:\mathbb{N})\\
	&\hspace{60mm}Rendeles_2=\textbf{rec}(nev:String, bevetel:\mathbb{N})\\
	&Ef=(x=x'~\wedge~|x'|>0~\wedge~x.azon\uparrow(x.ido\uparrow))\\
	&Uf=(y=\bigoplus_{e\in x'} <(bevetel,~elso.nev)=\sum\limits_{e\in x'~\wedge~elso.nev = e.nev}{(e.ar*e.adag)}~\wedge~\\
	&~\wedge~elso.nev\neq e.nev\rightarrow elso.nev=e.nev >)
	\end{flalign*}	
	A maximumkeresés:
	\begin{flalign*}
	&A=(y:Enor(Rendeles_2),~cout:String,~max:\mathbb{N},~elem:Rendeles_2)\\
	&Ef=(y=y'~\wedge~|y'|>0~\wedge~y.azon\uparrow)\\
	&Uf=((max,~elem)=\max_{e\in y'}{(e.bevetel)}~\wedge~cout=elem.nev)
	\end{flalign*}
	\newpage
	%% Absztrakt prgram fejezet %%
	\fejezet{Absztrakt~program}\\
	\begin{center}
	\begin{tabular}{|lll|}
		\hline
		\multicolumn{3}{|c|}{\textbf{Maximum~keresés}}\\
		\hline
		t & $\sim$ & \textit{Rendelések bevételeivel való felsorolója}\\
		+,0 & $\sim$~ & \textit{+,0}\\
		f(e) & $\sim$ & \textit{e.bevetel}\\
		\hline
	\end{tabular}

	\noindent\hfill
		\begin{stuki}[12cm]
			\stm{t.First()}
			\stm{max,~elem:=t.Current().bevetel,~t.Current()}
			\stm{t.Next()}
			\begin{WHILE}{4}{\stm{\lnot t.End()}}
				\begin{IF}[70]{2}{\stm{t.Current().bevetel>max}}
					\stm{max,~elem:=t.Current().bevetel,~t.Current()}
				\ELSE
					\stm{SKIP}
				\end{IF}
				\stm{t.Next()}
			\end{WHILE}
			\stm{cout:=elem.nev}
		\end{stuki}
		\vspace{5mm}
	Rendelések bevételeivel való felsorolója:
	\begin{tabular}{|l|l|}
	
	\end{tabular}

	\begin{tabular}{|lll|}
		\hline
		\multicolumn{3}{|c|}{\textbf{Összegzés~keresés}}\\
		\hline
		t & $\sim$ & \textit{Rendelések bevételeivel való felsorolója}\\
		+,0 & $\sim$~ & \textit{+,0}\\
		f(e) & $\sim$ & \textit{e.bevetel}\\
		\hline
	\end{tabular}
	\end{center}
\end{document}