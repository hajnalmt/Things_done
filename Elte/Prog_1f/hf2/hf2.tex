% Készítette: Hajnal Máté
% Az Elte Programtervező Informatikus szakához tartozó Programozás tantárgyhoz tartozó házi feladatom.
% A feladatok közül a 11-es

\documentclass[12pt,a4paper]{article}			%Article dokumentum
\usepackage[utf8]{inputenc}						%UTF-8-as kódolás
\usepackage{t1enc}								%Furcsa betűk
\usepackage{fancyhdr}							%For footer and header

%LANGUAGE-FONT
%texlive-lang-hungarian package should be installed!
\usepackage[english,magyar]{babel}				%Magyar nyelv 
\usepackage{mathptmx}							%Times New Roman font
\usepackage[shortlabels]{enumitem}							%For enumerations
\usepackage{url}								%For URL-s
\usepackage[headheight=56pt]{geometry}			%For the heading gap

%MATHEMATICAL EXPRESSIONS
\usepackage[fleqn]{amsmath}						%Mathematical expressions
\usepackage{amsfonts}							%Mathematical fonts
\usepackage{bm}									%Bolding

%TABLES, TABULATING, DRAWING
\usepackage{array}
\usepackage{tabu}
\usepackage{multirow}
\usepackage{tabularx}
\usepackage{tikz}
\usepackage{stuki}

%HEADER
\sloppy
\fancyhead{}
\fancyhead[C]{\textbf{1.beadandó feladat/11.feladat}}
\fancyhead[R]{\today}
\fancyhead[L]{Hajnal~Máté \newline RJBSCJ \newline \url{hajnalmt@inf.elte.hu} \newline 5.csoport}
\pagestyle{fancy}

%Fejezet stílus deklaráció
\newcommand{\fejezet}[1]{\noindent \textbf{\textit{\large #1 \vspace{5mm}}}}

%DOCUMENT
\begin{document}
	\pagenumbering{arabic}
	
	%% Feladat fejezet %%
	\fejezet{Feladat}\\
	\textit{Egy étteremben a pincérek által felvett rendeléseket egy szekvenciális input fájlban tartják nyilván az ételek neve, azon belül a rendelések időpontja szerint rendezett formában. Feltehetjük, hogy a fájl nem üres. A tárolt adatok: a rendelt étel neve, a rendelés időpontja, rendelt adagok száma, egy adag ára. Melyik étel hozta az étteremnek a legtöbb bevételt (összesített darab*egységár)?}
	\vspace{5mm}

	%% Specifikáció fejezet %%
	\fejezet{Specifikáció}\\
	A feladat állapottere többféleképpen felírható
	% A feladat állapottere
	\begin{flalign*}
		A=(f:Infile(&\mathbb{K}),~cout:Outfile(\mathbb{K}))\\
		A=(f:Infile(&Sor),~cout:Outfile(Sor))\\
		A=(f:Infile(&Rendeles), cout:String)\\
		&Rendeles=\textbf{rec}(nev:String, ido:\mathbb{N}, adag:\mathbb{N}, ar:\mathbb{N})\\
	\end{flalign*}
	Számunkra a legideálisabb rendelés nevű rekordokkal dolgozni egy felsorolóban. A feladat szempontjából az a legegyszerűbb ha két részfeladatban oldjuk meg. Az első részben egy szekvenciális input fájlban összegezzük minden ételre a bevételt, a másodikban megkeressük a maximumbevételt hozót. Tehát egy összegzésről és egy maximum keresésről van szó.  
	% Összegzés, összefűzés
	\begin{flalign*}
	A=(t:Infile(Rendeles_1),~cout:Outfile(&Rendeles_2))\\
	Rendeles_1=\textbf{rec}(nev:String,& ido:\mathbb{N}, adag:\mathbb{N}, ar:\mathbb{N})\\
	&Rendeles_2=\textbf{rec}(nev:String, bevetel:\mathbb{N})\\
	Ef=(t=t'~\wedge~|t'||>0~\wedge~t.azon\downarrow(t.ido\downarrow))\\
	Uf=((max, elem.nev)=\max_{e\in t'}{e.ar*e.adag})\\
	\end{flalign*}	
	
\end{document}