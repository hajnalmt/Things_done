%----------------------------------------------------------------------------------------
%   PACKAGES AND OTHER DOCUMENT CONFIGURATIONS
%----------------------------------------------------------------------------------------

\documentclass[11pt,a4paper,sans]{moderncv} 
\moderncvstyle{classic} % CV theme - options include: 'casual' (default), 'classic', 'oldstyle' and 'banking'
\moderncvcolor{blue} % CV color - options include: 'blue' (default), 'orange', 'green', 'red', 'purple', 'grey' and 'black'
\usepackage[utf8]{inputenc}
\usepackage{t1enc}
\usepackage[magyar]{babel}
\usepackage[scale=0.75]{geometry} % Reduce document margins
%\setlength{\hintscolumnwidth}{3cm} % Uncomment to change the width of the dates column
%\setlength{\makecvtitlenamewidth}{10cm} % For the 'classic' style, uncomment to adjust the width of the space allocated to your name
\include{logos}

%----------------------------------------------------------------------------------------
%   NAME AND CONTACT INFORMATION SECTION
%----------------------------------------------------------------------------------------

\firstname{Hajnal}
\lastname{Máté}
% All information in this block is optional, comment out any lines you don't need
\mobile{+3670 946 1305}
\email{hajnalmt@gmail.com}
\extrainfo{\githublogo \httplink{https://github.com/hajnalmt}\\ \linkedinlogo \httplink{https://www.linkedin.com/in/hajnalmt}\\ 1993/05/22}
\photo[80pt][0pt]{cv-image.jpg} % The first bracket is the picture height, the second is the thickness of the frame around the picture (0pt for no frame)
\quote{}

%----------------------------------------------------------------------------------------

\begin{document}
\makecvtitle % Print the CV title
%----------------------------------------------------------------------------------------
%   EDUCATION SECTION
%----------------------------------------------------------------------------------------
\vspace{-1.5cm}
\section{Tanulmányok}
\cventry{2016-}{BME Villamosmérnöki és Informatikai Kar}{}{MSC villamosmérnöki szak \newline Számítógép-alapú rendszerek főspecializáció \newline Alkalmazott elektronika mellékspecializáció \newline Önálló labor 1: Gantt diagramm alkalmazás HTML5-el}{}{}{}{}
\cventry{2011--2015}{BME Villamosmérnöki és Informatikai Kar}{}{BSC villamosmérnöki szak \newline Minden félévben tanulmányi kiemelt, 2013 második félévétől pedig közösségi kiemelt is 2012 második félévében a BSC villamosmérnökök között a legmagasabb ösztöndíj átlagot elért hallgató \newline 2013-tól az Automatizálási és Alkalmazott Informatikai Tanszéken a Beágyazott és Irányító rendszerek szakirány \newline Önálló laboratórium téma: DSP irányítású feszültségcsökkentő DCDC átalakító tervezése \newline Szakdolgozat téma: GraintAutLine – poligonok (márványszemcsék képének) konvex síkidomokra bontás}{}{}{}{}
\cventry{2014--}{ELTE Informatikai Kar}{}{BSC rogramtervező informatikus szak \newline Szoftverfejlesztő Informatikus szakirány}{}{}{}{}
\cventry{2005--2011}{Kecskeméti Református Gimnázium.}{}{\newline Érettségi Jeles eredménnyel, matek-fizika-informatika fakultáción \newline Országos Református Iskolák közötti matematika verseny 1. hely (2011) \newline Középiskolás Matematikai Lapok B pontverseny 12. osztályosok 4. hely (2011) \newline Szakács Jenő megyei fizikaverseny 2. hely (2011)}{}{}{}{}

%----------------------------------------------------------------------------------------
%   WORK EXPERIENCE SECTION
%----------------------------------------------------------------------------------------

\section{Munka tapasztalatok}

\cvitem{2015~szept.- 2016~március}{\textbf{Robert BOSCH Kft.} gyakornok \newline Linux build rendszer feltérképezése és dokumentálása}
\cvitem{2015 január-június}{\textbf{Conet Kft.} gyakornok \newline Szakmai gyakorlat: Stm32f105-ös mikrokontrollerre firmware írása}
\cvitem{2014--2015}{\textbf{Mad World} rendező \newline Egyetemi szintű szórakozóhely, ahol heti rendszerességgel 500-600 fős rendezvényeket bonyolítunk le}
\cvitem{2011 nyara}{\textbf{Weasler Kft.} nyári munka, CNC esztergagépek alapszintű kezelése}{}{}
\newpage

%----------------------------------------------------------------------------------------
%   AWARDS SECTION
%----------------------------------------------------------------------------------------

\section{Informatikai ismeretek}

\cvitem{}{Programozási ismeretek (C, C++, Assembly, Verilog)}
\cvitem{}{Linux operációs rendszer, Bash script, Powershell scriptírási alapismeretek}
\cvitem{}{Funkcionális nyelvi alapismeretek (Haskell, Erlang)}
\cvitem{}{Webfejlesztési alapismeretek (HTML, CSS, PHP, SQL, JAVASCRIPT)}
\cvitem{}{Microsoft Office programok magasszintű használata}
\cvitem{}{Matlab, Maple, Eclipse, Visual Studio, CodeBlocks, Qt alkalmazások ismerete}
\cvitem{}{Elektronikai műszerek használatában való jártasság}

%----------------------------------------------------------------------------------------
%   INTERESTS SECTION
%------------------------------------------------------------------------------

\section{Egyéb}
\cvitem{}{Angol nyelv: Középfokú komplex nyelvvizsga bizonyítvány}
\cvitem{}{Német nyelv: Középfokú komplex nyelvvizsga bizonyítvány}
\cvitem{}{A és B kategóriás jogosítvány}


%----------------------------------------------------------------------------------------
%   PERSONAL COMPETENCIES
%----------------------------------------------------------------------------------------

\section{Személyes kompetenciák}
\cvitem{}{Az elmúlt öt évben a folyamatos rendezvényszervezés és az egyetemi terhelés megtanított az időm hasznos és pontos beosztására a monotonitás- és stressztűrésre, valamint a feladataim megbízható és precíz végrehajtására. Rendkívül jó csapatjátékosnak tartom magam, és átalagon felülinek mondható a nyitottságom és a jó problémamegoldó készségem.}
\cvitem{}{Szabadidőm a családommal és a barátaimmal szeretem tölteni. Ezen kívül, szoktam sportolni (10 évig karatéztam, 1 danos mester vagyok, síelek, snowboardozok, rollerezek), sokat olvasok és utazok. Egyik kedvenc hobbim az animációs filmek nézese.}

\textit{Közéleti tevékenységek \vspace{2pt} \newline}
\small{
	\cvitem{2015} {Schönherz Qpa rendező, gazdaságis. Ez a BME-VIK 2015-ös, ~2500 fős Kari napjainak megrendezését jelenti.}
	\cvitem{2014} {Schönherz Qpa győztes, a Big Fluffy Bears csapat csapatkapitánya}
	\cvitem{2013-2015}{Dezső buli rendező: Öntevékeny Kör a Schönherz Kör a Schönherz Zoltán Kollégiumban, mely során egy disznótort rendezünk ~250 fő részvételével a kollégiumba.}
		\cvitem{2012-}{SSSL (Szent Schönherz Senior Lovagrend): Öntevékeny Kör a Schönherz Zoltán Kollégiumban, a gólyákkal való foglalkozás és a programjaik (Gólyatábor, Gólyahét, Gólyabál) megszervezése a fő profilunk.
		\begin{itemize}	
			\item{2012-2016 képző}
			\item{Képzés csoportvezető}
			\item{Gólyahét FőHR-es}
			\item{Gólyabál szárnysegéd}
		\end{itemize}\vspace{-15pt}}
	\cvitem{2012-2015}{Pizzásch: Öntevékeny Kör a Schönherz Zoltán Kollégiumban. Minden szerdán este megsütünk több mint 120 pizzát a kollégistáknak.}
	\cvitem{2012-2015}{ClubCeption: Öntevékeny Kör a Schönherz Zoltán Kollégiumban, a kollégiumon belüli szórakozóhely üzemeltetése, gazdasági feladatok ellátása.}
}

%----------------------------------------------------------------------------------------

\end{document}
              